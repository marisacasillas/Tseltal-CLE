\documentclass[floatsintext,man]{apa6}

\usepackage{amssymb,amsmath}
\usepackage{ifxetex,ifluatex}
\usepackage{fixltx2e} % provides \textsubscript
\ifnum 0\ifxetex 1\fi\ifluatex 1\fi=0 % if pdftex
  \usepackage[T1]{fontenc}
  \usepackage[utf8]{inputenc}
\else % if luatex or xelatex
  \ifxetex
    \usepackage{mathspec}
    \usepackage{xltxtra,xunicode}
  \else
    \usepackage{fontspec}
  \fi
  \defaultfontfeatures{Mapping=tex-text,Scale=MatchLowercase}
  \newcommand{\euro}{€}
\fi
% use upquote if available, for straight quotes in verbatim environments
\IfFileExists{upquote.sty}{\usepackage{upquote}}{}
% use microtype if available
\IfFileExists{microtype.sty}{\usepackage{microtype}}{}

% Table formatting
\usepackage{longtable, booktabs}
\usepackage{lscape}
% \usepackage[counterclockwise]{rotating}   % Landscape page setup for large tables
\usepackage{multirow}		% Table styling
\usepackage{tabularx}		% Control Column width
\usepackage[flushleft]{threeparttable}	% Allows for three part tables with a specified notes section
\usepackage{threeparttablex}            % Lets threeparttable work with longtable

% Create new environments so endfloat can handle them
% \newenvironment{ltable}
%   {\begin{landscape}\begin{center}\begin{threeparttable}}
%   {\end{threeparttable}\end{center}\end{landscape}}

\newenvironment{lltable}
  {\begin{landscape}\begin{center}\begin{ThreePartTable}}
  {\end{ThreePartTable}\end{center}\end{landscape}}




% The following enables adjusting longtable caption width to table width
% Solution found at http://golatex.de/longtable-mit-caption-so-breit-wie-die-tabelle-t15767.html
\makeatletter
\newcommand\LastLTentrywidth{1em}
\newlength\longtablewidth
\setlength{\longtablewidth}{1in}
\newcommand\getlongtablewidth{%
 \begingroup
  \ifcsname LT@\roman{LT@tables}\endcsname
  \global\longtablewidth=0pt
  \renewcommand\LT@entry[2]{\global\advance\longtablewidth by ##2\relax\gdef\LastLTentrywidth{##2}}%
  \@nameuse{LT@\roman{LT@tables}}%
  \fi
\endgroup}


  \usepackage{graphicx}
  \makeatletter
  \def\maxwidth{\ifdim\Gin@nat@width>\linewidth\linewidth\else\Gin@nat@width\fi}
  \def\maxheight{\ifdim\Gin@nat@height>\textheight\textheight\else\Gin@nat@height\fi}
  \makeatother
  % Scale images if necessary, so that they will not overflow the page
  % margins by default, and it is still possible to overwrite the defaults
  % using explicit options in \includegraphics[width, height, ...]{}
  \setkeys{Gin}{width=\maxwidth,height=\maxheight,keepaspectratio}
\ifxetex
  \usepackage[setpagesize=false, % page size defined by xetex
              unicode=false, % unicode breaks when used with xetex
              xetex]{hyperref}
\else
  \usepackage[unicode=true]{hyperref}
\fi
\hypersetup{breaklinks=true,
            pdfauthor={},
            pdftitle={Early language experience in a Tseltal Mayan village},
            colorlinks=true,
            citecolor=blue,
            urlcolor=blue,
            linkcolor=black,
            pdfborder={0 0 0}}
\urlstyle{same}  % don't use monospace font for urls

\setlength{\parindent}{0pt}
%\setlength{\parskip}{0pt plus 0pt minus 0pt}

\setlength{\emergencystretch}{3em}  % prevent overfull lines


% Manuscript styling
\captionsetup{font=singlespacing,justification=justified}
\usepackage{csquotes}
\usepackage{upgreek}

 % Line numbering
  \usepackage{lineno}
  \linenumbers


\usepackage{tikz} % Variable definition to generate author note

% fix for \tightlist problem in pandoc 1.14
\providecommand{\tightlist}{%
  \setlength{\itemsep}{0pt}\setlength{\parskip}{0pt}}

% Essential manuscript parts
  \title{Early language experience in a Tseltal Mayan village}

  \shorttitle{Early language experience in a Tseltal Mayan village}


  \author{Marisa Casillas\textsuperscript{1}, Penelope Brown\textsuperscript{1}, \& Stephen C. Levinson\textsuperscript{1}}

  % \def\affdep{{"", "", ""}}%
  % \def\affcity{{"", "", ""}}%

  \affiliation{
    \vspace{0.5cm}
          \textsuperscript{1} Max Planck Institute for Psycholinguistics  }

  \authornote{
    Correspondence concerning this article should be addressed to Marisa
    Casillas, P.O. Box 310, 6500 AH Nijmegen, The Netherlands. E-mail:
    \href{mailto:Marisa.Casillas@mpi.nl}{\nolinkurl{Marisa.Casillas@mpi.nl}}
  }


  \abstract{Daylong at-home audio recordings from 10 Tseltal Mayan children
(0;2--3;0) were analyzed for how often children engaged in verbal
interaction with others and whether their speech environment changed
with age, time of day, household size, and number of speakers present.
Tseltal children were infrequently directly spoken to, with most
directed speech coming from adults, and no increase with age. Most
directed speech came in the mornings or afternoons, and interactional
peaks took the form of \textasciitilde{}1-minute bursts of turn taking.
An initial analysis of children's vocal development suggested that,
despite relatively little directed speech, Tseltal children develop
early language skills on a similar timescale to Western children.
Multiple proposals for how Tseltal children might learn language
efficiently are discussed.}
  \keywords{Child-directed speech, linguistic input, non-WEIRD, vocal maturity, turn
taking, interaction, Mayan \\

    \indent Word count: 10225 (8549 not including references)
  }





\usepackage{amsthm}
\newtheorem{theorem}{Theorem}[section]
\newtheorem{lemma}{Lemma}[section]
\theoremstyle{definition}
\newtheorem{definition}{Definition}[section]
\newtheorem{corollary}{Corollary}[section]
\newtheorem{proposition}{Proposition}[section]
\theoremstyle{definition}
\newtheorem{example}{Example}[section]
\theoremstyle{definition}
\newtheorem{exercise}{Exercise}[section]
\theoremstyle{remark}
\newtheorem*{remark}{Remark}
\newtheorem*{solution}{Solution}
\begin{document}

\maketitle

\setcounter{secnumdepth}{0}



\section{Introduction}\label{intro}

A great deal of work in developmental language science revolves around
one central question: what kind of linguistic experience (and how much)
is needed to support first language acquisition? In pursuing this topic,
many researchers have fixed their sights on the speech addressed to
children. In several languages, child-directed speech (CDS, speech
designed for and directed toward a child recipient) has been
demonstrated to be distinct from adult-directed speech (ADS) in that it
is linguistically adapted for young listeners (e.g., Soderstrom, 2007),
interactionally rich (Bruner, 1983), preferred by infants (ManyBabies
Collaborative, 2017), and facilitates early word learning (Cartmill et
al., 2013; Hoff, 2003; Rowe, 2008; Weisleder \& Fernald, 2013).

However, the role of CDS in typical language development is less clear
once we take a broad view of the world's language learning environments.
In any given linguistic community, the vast majority of children acquire
the linguistic system and language behaviors needed for successful
communication in the context in which they are raised. In many cases,
prior ethnographic work suggests that successful adult-like
communicative competence is typically achieved without frequent CDS
(Brown, 2011; de León, 2011; Gaskins, 2006; Ochs \& Schieffelin, 1984).
If so, two important considerations arise: (1) while CDS is a powerful
driver of learning in some contexts, it is unlikely to be universally
fundamental for typical language development (Brown, 2014; Brown \&
Gaskins, 2014), and (2) we should do more to explore other types of
linguistic experience and other features of the learning environment
that allow children to extract the information they need to learn
language.

Past work on child language development in communities with reportedly
infrequent CDS (e.g., Brown, 2011; de León, 2011; Gaskins, 2006; Ochs \&
Schieffelin, 1984) has tended to use rich linguistic and ethnographic
methods that, while well-suited to characterizing language
socialization, lack the quantitative rigor that would otherwise enable
reproducible results derived from reasonably representative participant
samples (but see Shneidman \& Goldin-Meadow, 2012). This situation calls
for work that applies quantitative methods from developmental language
science in diverse ethnolinguistic contexts in order to build more
robust theories of language learning. In this paper we investigate the
language environment and early vocal development of 10 Tseltal Mayan
children growing up in a community where caregivers have been previously
reported to infrequently directly speak to young children (Brown, 1998,
2011, 2014). Our aims are to quantitatively ground these prior
qualitative claims in order to reason about the fundamental factors for
learning language in Tseltal Mayan (and similar) communities.

\subsection{Child-directed speech}\label{intro-cds}

Prior work, conducted primarily in Western contexts, has shown that the
amount of CDS children hear influences their language development; more
CDS is associated with faster-growing receptive and productive
vocabularies (e.g., Hart \& Risley, 1995; Hoff, 2003; Shneidman \&
Goldin-Meadow, 2012), faster lexical retrieval (Weisleder \& Fernald,
2013), and faster syntactic development (Huttenlocher, Waterfall,
Vasilyeva, Vevea, \& Hedges, 2010). Given that CDS is designed for a
child hearer, it is more likely than ADS or other-directed speech to
align with the child's attention, and may thereby facilitate early
language development. There are, however, a few caveats to the body of
work relating CDS quantity and language development.

First, while there is overwhelming evidence linking CDS quantity to
vocabulary size, links to grammatical development are more scant (but
see Brinchmann, Braeken, \& Lyster, 2019; Frank, Braginsky, Marchman, \&
Yurovsky, in preparation; Huttenlocher et al., 2010). While the
advantage of CDS for referential word learning is clear, it is less
obvious how it facilitates syntactic learning (Yurovsky, 2018). On the
other hand, there is a wealth of evidence that syntactic knowledge is
lexically specified (e.g., Lieven, Pine, \& Baldwin, 1997), and that,
crosslinguistically, children's vocabulary size is one of the most
robust predictors of their early syntactic development (Frank et al., in
preparation; Marchman, Martínez-Sussmann, \& Dale, 2004)---what is good
for the lexicon may also be good for syntax.

Second, most work on CDS \emph{quantity} (i.e., how often children hear
CDS) uses summary measures that average over the ebb and flow of the
recorded session. In reality, verbal behaviors are highly temporally
structured: infants' and adults' vocal behavior is clustered across
multiple time scales of daylong recordings (Abney, Smith, \& Yu, 2017),
and nouns and verbs are used within short bursts separated by long
periods across languages (Blasi, Schikowski, Moran, Pfeiler, \& Stoll,
in preparation). In fact, experimental work has shown that children
sometimes learn better from bursty exposure to words (Schwab \&
Lew-Williams, 2016). Recordings of children's language experience over
the course of entire waking days is critical for understanding the full
range and distribution of linguistic interactions included in children's
everyday language experiences (Soderstrom \& Wittebolle, 2013;
Tamis-LeMonda, Custode, Kuchirko, Escobar, \& Lo, 2018).

DERP DERP DERP

Third, prior work has typically focused on Western (primarily North
American) populations, limiting our ability to generalize effects of CDS
to children elsewhere (Brown \& Gaskins, 2014; Henrich, Heine, \&
Norenzayan, 2010; M. Nielsen, Haun, Kärtner, \& Legare, 2017). While we
gain valuable insight by looking at within-population variation, we can
more effectively find places where our assumptions break down by
studying language development in communities that diverge meaningfully
(linguistically and culturally) from those already well-studied.
Linguistic anthropologists working in non-Western communities have long
reported that caregiver-child interaction varies immensely from place to
place, but that, despite this variation, children appear to achieve
major communicative benchmarks (e.g., pointing, first words) on a
similar timescale (Brown, 2011, 2014; Brown \& Gaskins, 2014; Gaskins,
2006; Liszkowski, Brown, Callaghan, Takada, \& de Vos, 2012; Ochs \&
Schieffelin, 1984). These findings have had a limited impact on
mainstream theories of language development, partly due to a lack of
directly comparable methods (Brown, 2014; Brown \& Gaskins, 2014).

A number of recent or ongoing research projects have used standard
psycholinguistic methods to investigate language learning environments
in traditional, non-Western communities, with several substantiating the
claim that children in many parts of the world hear little CDS. Scaff,
Cristia, and colleagues (2017; in preparation) estimate, based on
daylong recordings, that Tsimane children (Bolivian lowlands;
forager-horticulturalist) hear approximately 4.8 minutes of CDS per hour
between ages 0;6 and 3;0 when considering all possible environmental
speech (Cristia et al., 2017; Scaff et al., in preparation; see also
Vogt, Mastin, and Schots (2015)). Shneidman and Goldin-Meadow (2012)
analyzed speech from one-hour at-home video recordings of children
between 1;0 and 3;0 in a Yucatec Mayan and a North American community.
Their analyses yielded four main findings: compared to the American
children, (a) Yucatec children heard many fewer utterances per hour, (b)
a much smaller proportion of the utterances they heard were
child-directed, (c) the proportion of utterances that were
child-directed increased dramatically with age, matching U.S. children's
CDS proportion by 3;0, and (d) most of the added CDS in the Yucatec
sample came from other children (e.g., older siblings/cousins). The
lexical diversity of the CDS that Yucatec Mayan children heard at 24
months---particularly from adult speakers---predicted their vocabulary
knowledge at 35 months, suggesting that CDS characteristics still play a
role in that context.

The current study addresses two of these three issues by using both
daylong audio recordings and standard measures of vocal development to
better understand the process of language learning in a Tseltal Mayan
community. DERP DERP DERP\ldots{} Non-uniform distribution of input
\ldots{} activity contexts \ldots{} Reliable estimate of CDS rate and
check on non-delayed development.

\subsection{Vocal maturity of spontaneous
speech}\label{vocal-maturity-of-spontaneous-speech}

Past ethnographic work has reported that, despite hearing little CDS,
children in some contexts show no evidence of language delay (e.g.,
Brown, 2011, 2014; Brown \& Gaskins, 2014; Liszkowski et al., 2012). We
test this claim by comparing Tseltal children's achievement of major
speech production milestones to those already known for Western
children. In so doing, we report on the \enquote{vocal maturity} of
Tseltal children's spontaneous speech. Our vocal maturity measure is
designed to capture the transition from (a) non-canonical babble to
canonical babble, (b) canonical babble to first words, and (c)
single-word utterances to multi-word utterances. This measure is, at
best, a coarse approximation of children's true linguistic abilities,
but it is an efficient means for getting a bird's eye view of children's
speech as it becomes more linguistically complex over the first three
years.

Importantly, children's vocal maturity may be more subject to
environmental factors as they grow older. The onset of canonical
babbling during the first year appears to be overall relatively stable
in response to variable language environments (e.g., Lee, Jhang, Relyea,
Chen, \& Oller, 2018; Oller, Eilers, Basinger, Steffens, \& Urbano,
1995; Oller, Eilers, Neal, \& Cobo-Lewis, 1998). That said, there is
variation in the precise onset age of canonical babble; one longitudinal
study showed an onset age range of 0;9 to 1;3 among children from a
relatively homogenous middle-class sample (McGillion et al., 2017). The
same study showed that the age of onset for canonical babble
significantly predicted the age of onset for first words. Once children
begin producing recognizable words, environmental effects become more
apparent; vocabulary size---even very early vocabulary---is known to be
sensitive to language environment factors such as maternal education and
birth order (Arriaga, Fenson, Cronan, \& Pethick, 1998; Frank et al., in
preparation). Early vocabulary size is also a robust cross-linguistic
predictor of later syntactic development, including the age at which a
child is likely to have begun combining words (Frank et al., in
preparation; Marchman et al., 2004).

Therefore, if we indeed find that Tseltal children hear relatively
little CDS, one might expect that the emergence of canonical babble
would occur around the same age as it does in Western children, but that
the emergence of single words and multi-word utterances---would diverge
from known middle-class Western norms.

\subsection{The current study}\label{intro-currentstudy}

We examined the early language experience of 10 Tseltal Mayan children
under age 3;0 using daylong photo-linked audio recordings. Prior
ethnographic work suggests that Tseltal caregivers do not frequently
directly speak to their children until the children themselves begin to
actively initiate verbal interactions (Brown, 2011, 2014). Nonetheless,
Tseltal children develop language with no apparent delays (Brown, 2011,
2014; Liszkowski et al., 2012; see also Pye, 2017). We provide more
details on the community and dataset in the
\protect\hyperlink{methods}{Methods section}. We analyzed two basic
measures of Tseltal children's language environments: (a) the quantity
of speech directed to them (TCDS; target-child-directed speech) and (b)
the quantity of other-directed speech (ODS; speech directed to anyone
but the target child). We also then coarsely outline children's
linguistic development using vocal maturity estimates from their
spontaneous vocalizations.

Based on prior work, we predicted that Tseltal Mayan children would be
infrequently directly addressed, that the amount of TCDS would increase
with age, that most TCDS would come from other children, and that,
despite this, their early vocal development would show no sign of delay
with respect to known Western onset benchmarks.

\hypertarget{methods}{\section{Methods}\label{methods}}

\subsection{Corpus}\label{methods-dataset}

The children in this dataset come from a small-scale, subsistence
farming community in the highlands of Chiapas (Southern Mexico). The
vast majority of children in the community grow up speaking Tseltal
monolingually at home. Nuclear families are typically organized into
patrlineal clusters of large, multi-generation households. Tseltal
children's language environments have previously been characterized as
non-child-centered and non-object-centered (Brown, 1998, 2011, 2014).
During their waking hours, infants are typically tied to their mother's
back while she goes about her work for the day. When not on their
mother's back, young children are often cared for by other family
members, especially older siblings. Typically, TCDS is limited until
children themselves begin to initiate interactions, usually around age
1;0. Interactional exchanges, when they do occur, are often brief or
non-verbal (e.g., object exchange routines) and take place within a
multi-participant context (Brown, 2014). Interactions tend to focus on
appropriate actions and responses (not on words and their meanings), and
young children are socialized to attend to the activities taking place
around them (see also de León, 2011; Rogoff, Paradise, Arauz,
Correa-Chávez, \& Angelillo, 2003). By age five, most children are
competent speakers who engage in daily chores and the caregiving of
their younger siblings. The Tseltal approach to caregiving is similar to
that described for other Mayan communities (e.g., de León, 2011;
Gaskins, 2000; Pye, 1986; Rogoff et al., 2003; Shneidman \&
Goldin-Meadow, 2012).

The current data come from (corpus name and references retracted for
review), which includes raw daylong recordings and other developmental
language data from more than 100 children under 4;0 across two
traditional indigenous communities: the Tseltal Mayan community
described here and a Papua New Guinean community described elsewhere
(reference retracted for review). This Tseltal corpus, primarily
collected in 2015, includes raw recordings from 55 children born to 43
mothers. The participating families typically only had 2 to 3 children
(median = 2; range = 1--9), due to the fact that they come from a young
subsample of the community (mothers: mean = 26.3 years; median = 25;
range = 16--43 and fathers: mean = 30; median = 27; range = 17---52).
Based on data from living children, we estimate that, on average,
mothers were 20 years old when they had their first child (median = 19;
range = 12--27), with a following average inter-child interval of 3
years (median = 2.8; range = 1--8.5). As a result, 28\% of the
participating families had two children under 4;0. Household size,
defined in our dataset as the number of people sharing a kitchen or
other primary living space, ranged between 3 and 15 people (mean = 7.2;
median = 7). Although 32.7\% of the target children are first-born, they
were rarely the only child in their household. Most mothers had finished
primary (37\%; 6 years of education) or secondary (30\%; 9 years of
education) school, with a few more having completed preparatory school
(12\%; 12 years of education) or some university-level training (2\%
(one mother); 16 years of education); the remainder (23\%) had no
schooling or did not complete primary school. All fathers had finished
primary school, with most completing secondary school (44\%) or
preparatory school (21\%), and two completing some university-level
training (5\%). To our knowledge at the time of recording, all children
were typically developing.

When possible, we collected dates of birth for children using a medical
record card typically provided by the local health clinic within two
weeks of birth. However, some children do not have this card and
sometimes cards are created long after a child's birth. We asked all
parents to also tell us the approximate date of birth of the child, the
child's age, and an estimate of the time between the child's birth and
creation of the medical record card. We used these multiple sources of
information to triangulate the child's most likely date of birth if the
medical record card appeared to be unreliable, following up for more
details from the families if necessary.

We used a novel combination of a lightweight stereo audio recorder
(Olympus WS-832) and wearable photo camera (Narrative Clip 1) fitted
with a fish-eye lens to track children's interactions over the course of
a 9--11-hour period at home in which the experimenter was not present.
Ambulatory children wore both devices at once (as shown in
\protect\hyperlink{fig1}{Figure 1}) while other children wore the
recorder in a onesie while their primary caregiver wore the camera on an
elastic vest. The camera was set to take photos at 30-second intervals
and was synchronized to the audio in post-processing to generate
snapshot-linked audio (media post-processing scripts at:
\url{https://github.com/retracted_for_review}). We used these recordings
to capture a wide range of the linguistic patterns children encounter as
they participate in different activities over the course of their day
(Bergelson, Amatuni, Dailey, Koorathota, \& Tor, 2018; Greenwood,
Thiemann-Bourque, Walker, Buzhardt, \& Gilkerson, 2011; Tamis-LeMonda et
al., 2018; Tamis-LeMonda, Kuchirko, Luo, Escobar, \& Bornstein, 2017).

\begin{figure}

{\centering \includegraphics[width=0.3\linewidth]{Tseltal-CLE_files/TseltalCLE-RecordingVest} 

}

\caption{The recording vest included an Olympus audio recorder in the front horizontal pocket and a miniature camera with a fish-eye lens on the shoulder strap.}\label{fig:fig1}
\end{figure}

\subsection{Data selection and annotation}\label{methods-samples}

Although the Tseltal corpus contains more than 500 hours of raw
photo-linked audio, very little of it is useful without adding manual
annotation. We estimated that we could fully transcribe approximately 10
hours of the corpus over the course of three 6-week field stays in the
village between 2015 and 2018, given full-time help from a native member
of the community on each trip. This estimate was approximately correct:
average exhaustive transcription time for one minute of audio was around
50 minutes, given that many clips featured overlapping multi-speaker
talk and/or significant background noise. Given this high cost of
annotation, we sampled clips in a way that would let us ask about
age-related changes in children's language experience, but with enough
data per child to generate accurate estimates of their individual speech
environments (see also retracted for review). Our solution was as
follows:

We chose 10 children's recordings based on maximal spread in child age
(0;0--3;0), child sex, and maternal education
(\protect\hyperlink{tab1}{Table 1}; all had native Tseltal-speaking
parents). We selected one hour's worth of non-overlapping clips for
transcription from each recording in the following order: nine randomly
selected 5-minute clips, five manually selected 1-minute top
\enquote{turn-taking} clips, five manually selected 1-minute top
\enquote{vocal activity} clips, and one manually selected 5-minute
extension of the best 1-minute clip (see \protect\hyperlink{fig2}{Figure
2} for an overview of sample distribution within the recordings). The
idea in creating these different subsamples was to measure properties of
(a) children's \emph{average} language environments (\enquote{Random}),
(b) their most \emph{input-dense} language environments
(\enquote{Turn-taking}), and (c) their most \emph{mature vocal behavior}
(\enquote{Vocal activity}).

\begin{table}[tbp]
\begin{center}
\begin{threeparttable}
\caption{\label{tab:tab1}Demographic overview of the 10 children whose recordings are sampled in the current study.}
\begin{tabular}{lllll}
\toprule
Age & \multicolumn{1}{c}{Sex} & \multicolumn{1}{c}{Mother's age} & \multicolumn{1}{c}{Level of maternal education} & \multicolumn{1}{c}{People in house}\\
\midrule
0;01.25 & M & 26 & none & 8\\
0;03.18 & M & 22 & preparatory & 9\\
0;05.29 & F & 17 & secondary & 15\\
0;07.15 & F & 24 & primary & 9\\
0;10.21 & M & 24 & secondary & 5\\
1;02.10 & M & 21 & none & 9\\
1;10.03 & F & 31 & preparatory & 9\\
2;02.25 & F & 17 & primary & 5\\
2;08.05 & F & 28 & secondary & 5\\
3;00.02 & M & 28 & primary & 6\\
\bottomrule
\end{tabular}
\end{threeparttable}
\end{center}
\end{table}

The turn-taking and high-activity clips were chosen by two trained
annotators (the first author and a student assistant) who listened to
each raw recording in its entirety at 1--2x speed while actively taking
notes about potentially useful clips. The first author then reviewed the
list of candidate clips and chose the best five 1-minute samples for
each of the two activity types. Note that, because the manually selected
clips did not overlap with the initial \enquote{random} clip selection,
the \enquote{true} peak turn-taking and vocal-activity clips for the day
could have possibly occurred during the random clips. High-quality
turn-taking activity was defined as closely timed sequences of
contingent vocalization between the target child and at least one other
person (i.e., frequent vocalization exchanges). High-quality vocal
activity clips were defined as periods in which the target child
produced the most and most diverse spontaneous (i.e., not imitative)
vocalizations (full instructions at
\url{https://git.io/retracted_for_review}).

\begin{figure}
\centering
\includegraphics{Tseltal-CLE_files/figure-latex/fig2-1.pdf}
\caption{\label{fig:fig2}Recording duration (black line) and sampled clips
(colored boxes) for each of the 10 recordings analyzed, sorted by child
age in months.}
\end{figure}

The 10 hours of clips were then transcribed and annotated by the first
author and a native speaker of Tseltal who personally knows all the
recorded families. Transcription was done in ELAN (Wittenburg, Brugman,
Russel, Klassmann, \& Sloetjes, 2006) using the ACLEW Annotation Scheme
(full documentation at \url{https://osf.io/b2jep/wiki/home/}, Casillas
et al., 2017). Utterance-level annotations included: an orthographic
transcription (Tseltal), a loose translation (Spanish), a vocal maturity
rating for each target child utterance (non-linguistic/non-canonical
babbling/canonical babbling/single words/multiple words), and the
intended addressee type for all non-target-child utterances
(target-child/other-child/adult/adult-and-child/animal/other-speaker-type).
Intended addressee was determined using contextual and interactional
information from the photos, audio, and preceding and following footage;
utterances with no clear intended addressee were marked as
\enquote{unsure}. We annotated lexical utterances as single- or
multi-word based on the word boundaries provided by the single native
speaker who reviewed all transcriptions; Tseltal is a mildly
polysynthetic language (words typically contain multiple morphemes).

\subsection{Data analysis}\label{methods-analysisinfo}

In what follows we first describe Tseltal children's speech environments
based on the nine randomly selected 5-minute clips from each child. We
investigate the effects of child age, time of day, household size, and
number of speakers on both TCDS min/hr and ODS min/hr. We then repeat
these analyses, only now looking at the high \enquote{turn-taking}
clips. Finally, we wrap up by outlining a coarse trajectory of Tseltal
children's early vocal development.

\subsection{Statistical models}\label{statistical-models}

All analyses were conducted in R with generalized linear mixed-effects
regressions using the glmmTMB package, and all plots were generated with
ggplot2 (M. E. Brooks et al., 2017; R Core Team, 2018; Wickham, 2009).
All data and analysis code can be found at
\url{https://github.com/retracted_for_review} (temporarily available as
an anonymous OSF repository:
\url{https://osf.io/9xd5u/?view_only=03a351c1172f4d17af9fce634aefb65e})
Notably, both speech environment measures are naturally restricted to
non-negative (0--infinity) values. This implicit boundary restriction at
zero causes the distributional variance of the measures to become
non-gaussian (i.e., with a long right tail). We handle this issue by
using a negative binomial linking function in the regression, which
estimates a dispersion parameter (in addition to the mean and variance)
that allows the model to more closely fit our non-negative,
overdispersed data (M. E. Brooks et al., 2017; Smithson \& Merkle,
2013). When, in addition to this, extra cases of zero were evident in
the distribution (e.g., TCDS min/hr was zero because the child was by
themselves), we also added a zero-inflation model to the regression. A
zero-inflation negative binomial regression creates two models: (a) a
binary model to evaluate the likelihood of none vs.~some presence of the
variable (e.g., no vs.~some TCDS) and (b) a count model of the variable
(e.g., \enquote{3} vs. \enquote{5} TCDS min/hr), using the negative
binomial distribution as the linking function. Alternative, gaussian
linear mixed-effects regressions with logged dependent variables are
available in the Supplementary Materials, but the results are broadly
similar to what we report here.

\section{Results}\label{results}

Our model predictors were as follows: child age (months), household size
(number of people), and number of non-target-child speakers present in
that clip, all centered and standardized, plus time of day at the start
of the clip (as a factor; \enquote{morning} = up until 11:00;
\enquote{midday} = 11:00--13:00; and \enquote{afternoon} = 13:00
onwards). In addition, the model inluded two-way interactions between
child age and: (a) the number of speakers present, (b) household size,
and (c) time of day. We also added a random effect of child. For the
zero-inflation models, we included the number of speakers present. We
only report significant effects in the main text; full model outputs are
available in the Supplementary Materials.

\begin{figure}
\centering
\includegraphics{Tseltal-CLE_files/figure-latex/fig3-1.pdf}
\caption{\label{fig:fig3}Estimates of TCDS min/hr (left) and ODS min/hr
(right) across the sampled age range. Each box plot summarizes the data
for one child from the randomly sampled clips (purple; solid) or the
turn taking clips (green; dashed). Bands on the linear trends show 95\%
confidence intervals.}
\end{figure}

\begin{figure}
\centering
\includegraphics{Tseltal-CLE_files/figure-latex/fig4-1.pdf}
\caption{\label{fig:fig4}Average CDS rates reported from at-home recordings
across various populations and ages, including urban (empty shape) and
rural or indigenous (filled shape) samples. Point size indicates the
number of children represented (range = 1--26). Data sources: Bergelson
et al. (2019) US/Canada; Shneidman (2010) US and Yucatec; Vogt et al.
(2015) Dutch, Mozambique urban and rural; Scaff et al. (in preparation)
Tsimane.}
\end{figure}

\subsection{Target-child-directed speech
(TCDS)}\label{target-child-directed-speech-tcds}

The children in our sample were directly spoken to for an average of
3.63 minutes per hour in the random sample (median = 4.08; range =
0.83--6.55; \protect\hyperlink{fig3}{Figure 3}). These estimates are
similar to those reported for Yucatec Mayan children (Shneidman \&
Goldin-Meadow, 2012), as illustrated in \protect\hyperlink{fig4}{Figure
4} (see Scaff et al. (in preparation) for more detailed cross-language
comparisons). Note that, to make this comparison, we have converted
Shneidman's (2010) utterance/hr estimates to min/hr using the median
Tseltal utterance duration for non-target child speakers (1029 msec),
motivated by the fact that Yucatec and Tseltal are related languages
spoken in comparably rural indigenous communities.

We modeled TCDS min/hr in the random clips with a zero-inflated negative
binomial regression. The rate of TCDS in the randomly sampled clips was
primarily affected by factors relating to the time of day (see
\protect\hyperlink{fig5}{Figure 5} for an overview of time-of-day
findings). The count model showed that the children were more likely to
hear TCDS in the mornings than around midday (B = 0.83, SD = 0.40, z =
2.09, p = 0.04), with no difference between morning and afternoon (p =
0.21) or midday and afternoon (p = 0.19). Time-of-day effects varied by
age: older children showed a stronger afternoon dip in TCDS.
Specifically, they were significantly more likely to hear TCDS at midday
(B = 0.85, SD = 0.38, z = 2.26, p = 0.02) and marginally more likely to
hear it in the morning (B = 0.57, SD = 0.30, z = 1.90, p = 0.06)
compared to the afternoons. Older target children were also
significantly more likely to hear TCDS when more speakers were present,
compared to younger children (B = 0.57, SD = 0.19, z = 2.95, p
\textless{} 0.01). There were no other significant effects in either the
count or the zero-inflation model.

\begin{figure}
\centering
\includegraphics{Tseltal-CLE_files/figure-latex/fig5-1.pdf}
\caption{\label{fig:fig5}Estimates of TCDS min/hr (left panels) and ODS
min/hr (right panels) across the recorded day in the random clips (top
panels) and turn-taking (bottom panels) clips. Each box plot summarizes
the data for children age 1;0 and younger (light) or age 1;0 and older
(dark) at the given time of day.}
\end{figure}

In contrast to findings from Shneidman and Goldin-Meadow (2012) on
Yucatec Mayan, most TCDS in the current data came from adult speakers
(mean = 80.61\%, median = 87.22\%, range = 45.90\%--100\%), with no
evidence that TCDS from \emph{other} children increases with target
child age (Spearman's \emph{rho} = -0.29; \emph{p} = 0.42).

\subsection{Other-directed speech
(ODS)}\label{other-directed-speech-ods}

Children heard an average of 21.05 minutes of ODS per hour in the random
sample (median = 17.80; range = 3.57--42.80): that is, nearly six times
as much speech as was directed to them, on average. We modeled ODS
min/hr in the random clips with a zero-inflated negative binomial
regression. The count model of ODS in the randomly selected clips
revealed that the presence of more speakers was strongly associated with
more ODS (B = 0.68, SD = 0.09, z = 7.29, p \textless{} 0.001). There
were an average of 3.44 speakers present other than the target child in
the randomly selected clips (median = 3; range = 0--10), more than half
of whom were typically adults.

Like TCDS, ODS was also strongly affected by time of day
(\protect\hyperlink{fig5}{Figure 5}), showing a dip around midday.
Compared to midday, target children were overall significantly more
likely to hear ODS in the mornings (B = 0.45, SD = 0.18, z = 2.49, p =
0.01) and marginally more likely to hear it in the afternoons (B = 0.33,
SD = 0.16, z = 1.99, p = 0.05), with no significant difference between
ODS rates in the mornings and afternoons (p = 0.41). As before, ODS rate
varied across the day by target child age: older children were
significantly more likely to hear ODS in the afternoon than at midday (B
= 0.42, SD = 0.17, z = 2.42, p = 0.02), with no significant differences
between afternoon and morning (p = 0.33) or midday and morning (p =
0.19). There were no other significant effects on ODS rate, and no
significant effects in the zero-inflation models.

\subsection{TCDS and ODS during interactional
peaks}\label{tcds-and-ods-during-interactional-peaks}

The estimates just given for TCDS and ODS are based on a random sample
of clips from the day; they represent baseline rates of speech in
children's environment. How much speech do children hear during the
interactional peaks the are distributed throughout the day? To answer
this question we repeated the same analyses of TCDS and ODS as above,
only this time using the high turn-taking clips in the sample instead of
the random ones (see the green/dashed summaries in
\protect\hyperlink{fig3}{Figures 3} and \protect\hyperlink{fig5}{5}).

Children heard much more TCDS in the turn-taking clips---13.28 min/hr
(nearly 4x the random sample rate; median = 13.65; range =
7.32--20.19)---while also hearing less ODS---11.93 min/hr (nearly half
the random sample rate; median = 10.18; range = 1.37--24.42).

We analyzed both TCDS and ODS rate with parallel models to those used
for the random sample, though this time we did not include a
zero-inflation model for TCDS given that the child was, by definition,
being directly addressed in these clips (i.e., there were no cases of
zero TCDS in the turn-taking sample). Overall, time-of-day effects were
weaker in the turn-taking sample. TCDS rate was not significantly
impacted by time-of-day at all. Meanwhile, ODS rated \emph{did} show a
significant time-of-day effect, but the dip in ODS came later in the day
than what we saw in the random sample (i.e., afternoon, not midday;
afternoon-vs.-midday: B = 0.61, SD = 0.29, z = 2.07, p = 0.04,
afternoon-vs.-morning: B = 0.61, SD = 0.25, z = 2.41, p \textless{}
0.01). Older children were also significantly more likely to hear ODS
around midday compared to the morning (midday-vs.-morning: B = 0.54, SD
= 0.30, z = 1.77, p = 0.08), but heard significantly less ODS overall
than younger children (B = -0.80, SD = 0.23, z = -3.43, p = 0). Finally,
whereas the number of speakers present had significantly impacted both
TCDS and ODS rate in the random sample, it only significantly impacted
ODS rate in the turn-taking sample (random sample: B = 0.63, SD = 0.10,
z = 6.44, p \textless{} 0.001). This result suggests that the number of
speakers is a robust predictor of ODS quantity across different
contexts. Full model outputs are available in the Supplementary
Materials.

In sum, our results provide compelling evidence in support of prior work
claiming that Tseltal children hear very little directly addressed
speech (Brown, 1998, 2011, 2014) and that their speech input is
non-uniformly distributed over the course of the day (Abney et al.,
2017; Blasi et al., in preparation). Do Tseltal children then show any
obvious evidence of delay in their early vocal development?

\subsection{Vocal maturity}\label{vocal-maturity}

We assessed whether the Tseltal children's vocalizations demonstrated
transitions from (a) non-canonical babble to canonical babble, (b)
canonical babble to first words, and (c) single-word utterances to
multi-word utterances, at approximately the same ages as would be
expected in a Western context. We generated descriptive statistics
(summarized in \protect\hyperlink{fig6}{Figure 6}) for the proportional
use of all linguistic vocalization types in the children's utterances
(non-canonical babble, canonical babble, single words, and multiple
words). These figures are based on all annotated vocalizations from the
random, turn-taking, and high vocal activity samples together (N = 4725
vocalizations). We had predicted that the emergence of canonical babble
would occur around the same age as it does in Western children, but that
the emergence of single words and multi-word utterances might
theoretically diverge from known middle-class Western norms if Tseltal
children indeed hear little CDS.

In fact, we find that Tseltal children's vocalizations closely resemble
the typical onset benchmarks established for Western speech development,
from canonical babble through first word combinations. Western children
have been shown to begin producing non-canonical babbling around 0;2,
with canonical babbling appearing sometime around 0;7, first words
around 1;0, and first multi-word utterances appearing just after 1;6
(Frank et al., in preparation; Kuhl, 2004; Pine \& Lieven, 1993; Slobin,
1970; Tomasello \& Brooks, 1999; Warlaumont, Richards, Gilkerson, \&
Oller, 2014).These benchmarks are mirrored in the Tseltal children's
vocalizations, which are summarized in \protect\hyperlink{fig6}{Figure
6}: there is a decline in the use of non-canonical babble and an
accompanying increase in the use of canonical babble between 0;6 and
1;0; recognizable words are observed for every child from age 11;0 and
older; and multi-word utterances appear in all recordings at 1;2 and
later, making up 45\% of the oldest child's (3;0) vocalizations.

\begin{figure}
\centering
\includegraphics{Tseltal-CLE_files/figure-latex/fig6-1.pdf}
\caption{\label{fig:fig6}Proportion of vocalization types used by children
across age (NCB = Non-canonical babble, CB = Canonical babble, SW =
single word utterance, MW = multi-word utterance).}
\end{figure}

\subsubsection{Frequency of
vocalizations}\label{frequency-of-vocalizations}

We can use these same data to try and infer how \emph{often} children
use speech-like vocalizations (i.e., \enquote{usage} instead of
\enquote{onset} measures) (Warlaumont et al. (2014); retracted for
review). Between 2 and 14 months, the Tseltal children demonstrated a
large increase in the proportion of speech-like vocalizations (canonical
babbling and lexical speech): from 9\% before 0;6 to 58\% between 0;10
and 1;2. There is limited daylong data already published with which we
can compare these patterns, but we see that around age 1;0, the Tseltal
children's use of speech-like vocalizations (58\%) is nearly identical
to that reported by Warlaumont et al. (2014) for American children
around age 1;0 in an socioeconomically diverse sample (approximately
60\%). Futher, in a separate study, a subset of these Tseltal
vocalizations have been independently re-annotated and compared to
vocalizations from children acquiring five other non-related languages,
with very similar results: the ratio of speech-like vocalizations to all
linguistic vocalizations (canonical babbling ratio: Lee et al., 2018;
Oller \& Eilers, 1988) increases similarly under a variety of different
linguistic and childrearing environments between ages 0;2 and 3;0,
during which time children in all six communities begin to produce their
first words and multi-word utterances (retracted for review).

We also found that, in general, the Tseltal children did not vocalize
very often: they produced an average of 7.88 vocalizations per minute
(median = 7.55; range = 4.08--12.55) during their full one hour of
annotated audio (including the high vocal activity minutes), much of
which was crying and laughter. This rate is consistent with prior
estimates for the frequency of child-initiated prompts in Tseltal
interaction (Brown, 2011). Given that our age range goes all the way up
to 3;0, this rate is perhaps lower than what would be expected from past
work on recordings made in the lab (Oller et al., 1995; Oller, Eilers,
Steffens, Lynch, \& Urbano, 1994), in which 6--9 vocalizations per
minute was already evident at 16 months across a socioeconomically
diverse sample of U.S. participants. This finding would then appear to
be in-line with the idea that rate of vocalization is sensitive to the
language environment (Oller et al., 1995, 1994; Warlaumont et al.,
2014). That said, vocalization rate estimates from daylong recordings
would be necessary to more validly compare overall vocalization rates in
this case.

\section{Discussion}\label{disc}

We analyzed 10 Tseltal Mayan children's speech environments in order to
estimate how often they have the opportunity to attend and respond to
speech. Based on prior work, we predicted infrequent, but bursty use of
TCDS, an increase in TCDS with age, that a large proportion of TCDS
would come from other children, and that vocal development would be on
par with typically developing Western children. Only some of these
predictions were borne out in the analyses. We did find evidence for
infrequent use of TCDS and for a typical-looking trajectory of vocal
development, but we also found that most directed speech came from
adults, and that the quantity of directed speech was stable across the
first three years of life. Within individual recordings, TCDS and
contingent responding were influenced by the time of day and number of
speakers present. That said, time of day and number of speakers less
strongly impacted TCDS during high turn-taking clips, suggesting that
interactional peaks are one source of stable, high-engagement linguistic
experience available to Tseltal children in the first few years of life.
These findings only partly replicate estimates of child language input
and development in previous work on Yucatec Mayan and Tseltal Mayan
communities (Yucatec: Shneidman \& Goldin-Meadow 2012; Tseltal: Brown,
1998, 2011, 2014), and bring new questions to light regarding the
distribution of child-directed speech over activities and interactant
types in Mayan children's speech environments.

\subsection{Robust learning with less child-directed
speech}\label{robust-learning-with-less-child-directed-speech}

The bulk of our analyses were aimed at understanding how much speech
Tseltal children hear: we wanted to know how often they were directly
spoken to and how often they might have been able to listen to speech
directed to others. Consistent with prior work, the children were only
infrequently directly spoken to: an average of 3.63 minutes per hour in
the random sample. This average TCDS rate for Tseltal is approximately a
third of that found for North American children (Bergelson et al.,
2019), but is comparable to that for Tsimane children (Scaff et al., in
preparation) and Yucatec Mayan children (Shneidman \& Goldin-Meadow,
2012) in a similar age range. Meanwhile, we found that the children had
an enormous quantity of other-directed speech in their environment,
averaging 21.05 minutes per hour in the random sample, which is more
than has been previously reported for other cultural settings (e.g.,
Bergelson et al., 2019; Scaff et al., in preparation).

In sum, our daylong recording results confirm prior claims that Tseltal
children, like other Mayan children, are not often directly spoken to.
When they are, much of their speech comes in interactional sequences in
which children only play a minor part---directly contingent turn
transitions between children and their interlocutors are relatively
rare. However, we coarsely estimate that the typical child under age 3;0
experiences nearly two cumulative hours of high-intensity contingent
interaction with TCDS per day. If child-directed speech quantity
linearly feeds language development (such that more input begets more
(advanced) output), then the estimates presented here would lead us to
expect Tseltal to be delayed in their language development. However, our
analyses suggest that Tseltal children demonstrate vocal maturity
comparable to children from societies in which TCDS is known to be more
frequent (Frank et al., in preparation; Kuhl, 2004; Pine \& Lieven,
1993; Slobin, 1970; Tomasello \& Brooks, 1999; Warlaumont et al., 2014).
How might Tseltal children manage this feat?

\subsubsection{Other-directed speech}\label{other-directed-speech}

One proposal is that Mayan children become experts at learning from
observation during their daily interactions (de León, 2011; Rogoff et
al., 2003; Shneidman, 2010; Shneidman \& Goldin-Meadow, 2012). In the
randomly selected clips, children were within hearing distance of
other-directed speech for an average of 21.05 minutes per hour. This
large quantity of ODS is likely due to the fact that Tseltal children
tend to live in households with more people compared to North American
children (Shneidman \& Goldin-Meadow, 2012). In our data, the presence
of more speakers was associated with significantly more other-directed
speech, both based on the number of individual voices present in the
clip and on the number of people living in the household (for younger
children). The presence of more speakers had no overall impact on the
quantity of TCDS children experienced, but older children were more
likely than younger children to hear TCDS when more speakers were
present. These findings ring true with Brown's (2011, 2014) claim that
this Tseltal community is a non-child-centric; the presence of more
people primarily increases talk amongst the other speakers (i.e., not to
young children). But, as children become more sophisticated language
users, they are more likely to participate in others' talk. However,
given that an increase in the number of speakers is also likely
associated with an increase in the amount of overlapping speech, we
suggest that attention to ODS is unlikely to be the primary mechanism
underlying the robustness of early vocal development in Tseltal.
However, just because speech is hearable does not mean the children are
attending to it. Follow-up work on the role of ODS in language
development must better define what constitutes likely \enquote{listened
to} speech by the child.

\subsubsection{Increased TCDS with age}\label{increased-tcds-with-age}

Another possibility is that speakers more frequently address children
who are more communicatively competent (i.e., increased TCDS with age,
e.g., Warlaumont et al., 2014). In their longitudinal study of Yucatec
Mayan children, Shneidman and Goldin-Meadow (2012) found that TCDS
increased significantly with age, though most of the increase came from
other children speaking to the target child. Their finding is consistent
with other reports that Mayan children are more often cared for by their
older siblings from later infancy onward (2011, 2014). In our data,
there was no evidence for an overall increase in TCDS with age, neither
from adult speakers nor from child speakers. This non-increase in TCDS
with age may be due to the fact TCDS from other children was overall
infrequent in our data, possibly because: (a) the children were
relatively young and so spent much of their time with their mothers, (b)
these particular children did not have many older siblings, and (c) in
the daylong recording context more adults were present to talk to each
other than would be typical in a short-format recording (as used in
Shneidman \& Goldin-Meadow, 2012). That aside, we conclude from these
findings, that an increase in TCDS with age is also unlikely to explain
the robust pattern of Tseltal vocal development.

\subsubsection{Learning during interactional
bursts}\label{learning-during-interactional-bursts}

A third possibility is that children learn effectively from short,
routine language encounters. Bursty input appears to be the norm across
a number of linguistic and interactive scales (e.g., Abney et al., 2017;
Blasi et al., in preparation), and experiment-based work suggests that
children can benefit from massed presentation of new information (Schwab
\& Lew-Williams, 2016). We propose two mechanisms through which Tseltal
children might capitalize on the distribution of speech input in their
environment: (a) they experience most language input during routine
activities and (b) they consolidate their language experiences during
the downtime between interactive peaks. Neither of these mechanisms are
proposed to be particular to Tseltal children, but might be employed to
explain their efficient learning.

Tseltal children's linguistic input is not uniformly distributed over
the day: children were most likely to encounter speech, particularly
directed, contingent speech in the mornings and late afternoons,
compared to midday. Older children, who are less often carried and were
therefore more free to seek out interactions, showed these time of day
effects most strongly, eliciting TCDS both in the mornings (when the
entire household is present) and around midday (when many have dispersed
for farming or other work). An afternoon dip in environmental speech,
similar to what we report here, has been previously found for North
American children's daylong recordings (Greenwood et al., 2011;
Soderstrom \& Wittebolle, 2013). The presence of a similar effect in
Tseltal suggests that non-uniform distributions of linguistic input may
be the norm for children in a variety of different cultural-economic
contexts. Our findings here are the first to show that those time of day
effects change with age in the first few years across a number of speech
environment features (TCDS, TC--O transitions, O--TC transitions, and
(marginally) ODS). These time of day effects likely arise from the
activities that typically occur in the mornings and late
afternoons---meal preparation and dining in particular---while short
bouts of sleep could contribute to the afternoon dip (Soderstrom \&
Wittebolle, 2013). That said, in data from North American children
(Soderstrom \& Wittebolle, 2013), the highest density speech input came
during storytime and organized playtime (e.g., sing-alongs, painting),
while mealtime was associated with less speech input. We expect that
follow-up research tracking TCDS during activities in the Tseltal data
will lead to very different conclusions: storytime and organized
playtime are vanishingly rare in this non-child-centric community, and
mealtime may represent a time of routine and rich linguistic experience.
In both cases, however, the underlying association with activity (not
hour) implies a role for action routines that help children optimally
extract information about what words, agents, objects, and actions they
will encounter and what they are expected to do in response (see, e.g.,
Bruner, 1983; Tamis-LeMonda et al., 2018).

A more speculative possibility is that Tseltal children learn language
on a natural input-consolidation cycle: the rarity of interactional
peaks throughout the day may be complemented by an opportunity to
consolidate new information. Sleep has been shown to benefit language
learning tasks in both adults (Frost \& Monaghan, 2017; Mirković \&
Gaskell, 2016) and children (Gómez, Bootzin, \& Nadel, 2006; Horváth,
Liu, \& Plunkett, 2016; Hupbach, Gómez, Bootzin, \& Nadel, 2009),
including word learning, phonotactic constraints, and syntactic
structure. Our impression, both from the recordings and informal
observations made during visits to the community, is that young Tseltal
children frequently sleep for short periods throughout the day,
particularly at younger ages when they spend much of their day wrapped
within the shawl on their mother's back. Mayan children tend to pick
their own resting times; there are no formalized \enquote{sleep} times,
even at bedtime (Morelli, Rogoff, Oppenheim, \& Goldsmith, 1992), and
Mayan mothers take special care to keep infants in a calm and soothing
environment in the first few months of life (e.g., de León, 2011; Pye,
1986). There is little quantitative data on Mayan children's daytime and
nighttime sleeping patterns, but one study estimates that Yucatec Mayan
children between 0;0 and 2;0 sleep or rest nearly 15\% of the time
between morning and evening (Gaskins, 2000), doing so at times that
suited the child (Morelli et al., 1992). If Tseltal children's
interactional peaks are bookended by short sleeping periods, it could
contribute to efficient consolidation of new information encountered.
How often Tseltal children sleep, how deeply, and how their sleeping
patterns may relate to their linguistic development is an important
topic for future research.

\subsection{Limitations and Future Work}\label{disc-limfut}

The current findings are based on a cross-sectional analysis of 600
annotated recording minutes, divided among only ten children. The data
are limited mainly to verbal activity; we cannot analyze gaze and
gestural behavior. We have also used overall vocal maturity as an index
of language development, but further work should include receptive and
productive measures of linguistic skill with both experiment- and
questionnaire- based measures, as well as more in-depth analyses of
children's spontaneous speech, building on past work (Brown, 1998, 2011,
2014; Brown \& Gaskins, 2014). In short, more and more diverse data are
needed to enrich this initial description of Tseltal children's language
environments. Importantly, the current analyses are based on a corpus
that is still under active development: as new data are added,
up-to-date versions of these analyses will be available with the current
data and analysis scripts at:
\url{https://retracted_for_review.shinyapps.io/retracted_for_review/}.

\subsection{Conclusion}\label{disc-conclusion}

We estimate that, over the course of a waking day, Tseltal children
under age 3;0 hear an average of 3.63 minutes of directed speech per
hour. Contingent turn taking tends to occur in sparsely distributed
bursts often with a dip in the mid- to late-afternoon, particularly for
older children. Tseltal children's vocal maturity is on track with prior
estimates from populations in which child-directed speech is much more
frequent, raising a challenge for future work: how do Tseltal children
efficiently extract information from their linguistic environments? In
our view, a promising avenue for continued research is to more closely
investigate how directed speech is distributed over activities over the
course of the day and to explore a possible input-consolidation cycle
for language exposure in early development. By better understanding how
Tseltal children learn language, we hope to help uncover how human
language learning mechanisms are adaptive to the many thousands of
ethnolinguistic environments in which children develop.

\section{Acknowledgements}\label{acknowledgements}

Retracted for review

\newpage

\section{References}\label{refs}

\begingroup
\setlength{\parindent}{-0.5in} \setlength{\leftskip}{0.5in}

\hypertarget{refs}{}
\hypertarget{ref-abney2017time}{}
Abney, D. H., Smith, L. B., \& Yu, C. (2017). It's time: Quantifying the
relevant time scales for joint attention. In G. Gunzelmann, A. Howes, T.
Tenbrink, \& E. Davelaar (Eds.), \emph{Proceedings of the 39th Annual
Meeting of the Cognitive Science Society} (pp. 1489--1494). London, UK.

\hypertarget{ref-arriaga1998scores}{}
Arriaga, R. I., Fenson, L., Cronan, T., \& Pethick, S. J. (1998). Scores
on the macarthur communicative development inventory of children from
low- and middle-income families. \emph{Applied Psycholinguistics},
\emph{19}(2), 209--223.

\hypertarget{ref-bergelson2018day}{}
Bergelson, E., Amatuni, A., Dailey, S., Koorathota, S., \& Tor, S.
(2018). Day by day, hour by hour: Naturalistic language input to
infants. \emph{Developmental Science}, \emph{22}(1), e12715.
doi:\href{https://doi.org/10.1111/desc.12715}{10.1111/desc.12715}

\hypertarget{ref-bergelsoncasillas2019what}{}
Bergelson, E., Casillas, M., Soderstrom, M., Seidl, A., Warlaumont, A.
S., \& Amatuni, A. (2019). What do North American babies hear? A
large-scale cross-corpus analysis. \emph{Developmental Science},
\emph{22}(1), e12724.
doi:\href{https://doi.org/10.1111/desc.12724}{10.1111/desc.12724}

\hypertarget{ref-blasiIPhuman}{}
Blasi, D., Schikowski, R., Moran, S., Pfeiler, B., \& Stoll, S. (in
preparation). Human communication is structured efficiently for first
language learners: Lexical spikes.

\hypertarget{ref-brinchmann2019direct}{}
Brinchmann, E. I., Braeken, J., \& Lyster, S.-A. H. (2019). Is there a
direct relation between the development of vocabulary and grammar?
\emph{Developmental Science}, \emph{22}(1), e12709.
doi:\href{https://doi.org/10.1111/desc.12709}{10.1111/desc.12709}

\hypertarget{ref-brooks2017modeling}{}
Brooks, M. E., Kristensen, K., van Benthem, K. J., Magnusson, A., Berg,
C. W., Nielsen, A., \ldots{} Bolker, B. M. (2017). Modeling
zero-inflated count data with glmmTMB. \emph{bioRxiv}.
doi:\href{https://doi.org/10.1101/132753}{10.1101/132753}

\hypertarget{ref-brown1998conversational}{}
Brown, P. (1998). Conversational structure and language acquisition: The
role of repetition in Tzeltal adult and child speech. \emph{Journal of
Linguistic Anthropology}, \emph{2}, 197--221.
doi:\href{https://doi.org/10.1525/jlin.1998.8.2.197}{10.1525/jlin.1998.8.2.197}

\hypertarget{ref-brown2011cultural}{}
Brown, P. (2011). The cultural organization of attention. In A. Duranti,
E. Ochs, \& and B. B. Schieffelin (Eds.), \emph{Handbook of Language
Socialization} (pp. 29--55). Malden, MA: Wiley-Blackwell.

\hypertarget{ref-brown2014interactional}{}
Brown, P. (2014). The interactional context of language learning in
Tzeltal. In I. Arnon, M. Casillas, C. Kurumada, \& B. Estigarribia
(Eds.), \emph{Language in interaction: Studies in honor of Eve V. Clark}
(pp. 51--82). Amsterdam, NL: John Benjamins.

\hypertarget{ref-brown2014language}{}
Brown, P., \& Gaskins, S. (2014). Language acquisition and language
socialization. In N. J. Enfield, P. Kockelman, \& J. Sidnell (Eds.),
\emph{Handbook of Linguistic Anthropology} (pp. 187--226). Cambridge,
UK: Cambridge University Press.
doi:\href{https://doi.org/10.1017/CBO9781139342872.010}{10.1017/CBO9781139342872.010}

\hypertarget{ref-bruner1983childs}{}
Bruner, J. (1983). \emph{Child's talk}. Oxford: Oxford University Press.
doi:\href{https://doi.org/10.1177/026565908500100113}{10.1177/026565908500100113}

\hypertarget{ref-cartmill2013quality}{}
Cartmill, E. A., Armstrong, B. F., Gleitman, L. R., Goldin-Meadow, S.,
Medina, T. N., \& Trueswell, J. C. (2013). Quality of early parent input
predicts child vocabulary 3 years later. \emph{Proceedings of the
National Academy of Sciences}, \emph{110}(28), 11278--11283.
doi:\href{https://doi.org/10.1073/pnas.1309518110}{10.1073/pnas.1309518110}

\hypertarget{ref-casillas2017ACLEWDAS}{}
Casillas, M., Bunce, J., Soderstrom, M., Rosemberg, C., Migdalek, M.,
Alam, F., \ldots{} Garrison, H. (2017). Introduction: The ACLEW DAS
template {[}training materials{]}. Retrieved from
\url{https://osf.io/aknjv/}

\hypertarget{ref-cristia2017child}{}
Cristia, A., Dupoux, E., Gurven, M., \& Stieglitz, J. (2017).
Child-directed speech is infrequent in a forager-farmer population: A
time allocation study. \emph{Child Development}, \emph{Early View},
1--15. doi:\href{https://doi.org/10.1111/cdev.12974}{10.1111/cdev.12974}

\hypertarget{ref-deleon2011language}{}
de León, L. (2011). Language socialization and multiparty participation
frameworks. In A. Duranti, E. Ochs, \& and B. B. Schieffelin (Eds.),
\emph{Handbook of Language Socialization} (pp. 81--111). Malden, MA:
Wiley-Blackwell.
doi:\href{https://doi.org/10.1002/9781444342901.ch4}{10.1002/9781444342901.ch4}

\hypertarget{ref-frankIPvariability}{}
Frank, M. C., Braginsky, M., Marchman, V. A., \& Yurovsky, D. (in
preparation). \emph{Variability and consistency in early language
learning: The Wordbank project}. Retrieved from
\url{https://langcog.github.io/wordbank-book/}

\hypertarget{ref-frost2017sleep}{}
Frost, R. L. A., \& Monaghan, P. (2017). Sleep-driven computations in
speech processing. \emph{PloS One}, \emph{12}(1), e0169538.
doi:\href{https://doi.org/10.1371/journal.pone.0169538}{10.1371/journal.pone.0169538}

\hypertarget{ref-gaskins2000childrens}{}
Gaskins, S. (2000). Children's daily activities in a Mayan village: A
culturally grounded description. \emph{Cross-Cultural Research},
\emph{34}(4), 375--389.
doi:\href{https://doi.org/10.1177/106939710003400405}{10.1177/106939710003400405}

\hypertarget{ref-gaskins2006cultural}{}
Gaskins, S. (2006). Cultural perspectives on infant--caregiver
interaction. In N. J. Enfield \& S. Levinson (Eds.), \emph{Roots of
Human Sociality: Culture, Cognition and Interaction} (pp. 279--298).
Oxford: Berg.

\hypertarget{ref-gomez2006naps}{}
Gómez, R. L., Bootzin, R. R., \& Nadel, L. (2006). Naps promote
abstraction in language-learning infants. \emph{Psychological Science},
\emph{17}(8), 670--674.
doi:\href{https://doi.org/10.1111/j.1467-9280.2006.01764.x}{10.1111/j.1467-9280.2006.01764.x}

\hypertarget{ref-greenwood2011assessing}{}
Greenwood, C. R., Thiemann-Bourque, K., Walker, D., Buzhardt, J., \&
Gilkerson, J. (2011). Assessing children's home language environments
using automatic speech recognition technology. \emph{Communication
Disorders Quarterly}, \emph{32}(2), 83--92.
doi:\href{https://doi.org/10.1177/1525740110367826}{10.1177/1525740110367826}

\hypertarget{ref-hart1995meaningful}{}
Hart, B., \& Risley, T. R. (1995). \emph{Meaningful Differences in the
Everyday Experience of Young American Children}. Paul H. Brookes
Publishing.

\hypertarget{ref-henrich2010beyond}{}
Henrich, J., Heine, S. J., \& Norenzayan, A. (2010). Beyond WEIRD:
Towards a broad-based behavioral science. \emph{Behavioral and Brain
Sciences}, \emph{33}(2--3), 111--135.
doi:\href{https://doi.org/10.1017/S0140525X10000725}{10.1017/S0140525X10000725}

\hypertarget{ref-hoff2003specificity}{}
Hoff, E. (2003). The specificity of environmental influence:
Socioeconomic status affects early vocabulary development via maternal
speech. \emph{Child Development}, \emph{74}(5), 1368--1378.
doi:\href{https://doi.org/10.3389/fpsyg.2015.01492}{10.3389/fpsyg.2015.01492}

\hypertarget{ref-horvath2016daytime}{}
Horváth, K., Liu, S., \& Plunkett, K. (2016). A daytime nap facilitates
generalization of word meanings in young toddlers. \emph{Sleep},
\emph{39}(1), 203--207.
doi:\href{https://doi.org/10.5665/sleep.5348}{10.5665/sleep.5348}

\hypertarget{ref-hupbach2009nap}{}
Hupbach, A., Gómez, R. L., Bootzin, R. R., \& Nadel, L. (2009).
Nap-dependent learning in infants. \emph{Developmental Science},
\emph{12}(6), 1007--1012.
doi:\href{https://doi.org/10.1111/j.1467-7687.2009.00837.x}{10.1111/j.1467-7687.2009.00837.x}

\hypertarget{ref-huttenlocher2010sources}{}
Huttenlocher, J., Waterfall, H., Vasilyeva, M., Vevea, J., \& Hedges, L.
V. (2010). Sources of variability in children's language growth.
\emph{Cognitive Psychology}, \emph{61}(4), 343--365.
doi:\href{https://doi.org/10.1016/j.cogpsych.2010.08.002}{10.1016/j.cogpsych.2010.08.002}

\hypertarget{ref-kuhl2004early}{}
Kuhl, P. K. (2004). Early language acquisition: Cracking the speech
code. \emph{Nature Reviews Neuroscience}, \emph{5}(11), 831.
doi:\href{https://doi.org/10.1038/nrn1533}{10.1038/nrn1533}

\hypertarget{ref-lee2018babbling}{}
Lee, C.-C., Jhang, Y., Relyea, G., Chen, L.-m., \& Oller, D. K. (2018).
Babbling development as seen in canonical babbling ratios: A
naturalistic evaluation of all-day recordings. \emph{Infant Behavior and
Development}, \emph{50}, 140--153.

\hypertarget{ref-lieven1997lexically}{}
Lieven, E. V. M., Pine, J. M., \& Baldwin, G. (1997). Lexically-based
learning and early grammatical development. \emph{Journal of Child
Language}, \emph{24}(1), 187--219.
doi:\href{https://doi.org/10.1017/S0305000996002930}{10.1017/S0305000996002930}

\hypertarget{ref-liszkowski2012prelinguistic}{}
Liszkowski, U., Brown, P., Callaghan, T., Takada, A., \& de Vos, C.
(2012). A prelinguistic gestural universal of human communication.
\emph{Cognitive Science}, \emph{36}(4), 698--713.
doi:\href{https://doi.org/10.1111/j.1551-6709.2011.01228.x}{10.1111/j.1551-6709.2011.01228.x}

\hypertarget{ref-manybabies2017}{}
ManyBabies Collaborative. (2017). Quantifying sources of variability in
infancy research using the infant-directed speech preference.
\emph{Advances in Methods and Practices in Psychological Science},
1--46.
doi:\href{https://doi.org/10.31234/osf.io/s98ab}{10.31234/osf.io/s98ab}

\hypertarget{ref-marchman2004language}{}
Marchman, V. A., Martínez-Sussmann, C., \& Dale, P. S. (2004). The
language-specific nature of grammatical development: Evidence from
bilingual language learners. \emph{Developmental Science}, \emph{7}(2),
212--224.
doi:\href{https://doi.org/10.1111/j.1467-7687.2004.00340.x}{10.1111/j.1467-7687.2004.00340.x}

\hypertarget{ref-mcgillion2017paves}{}
McGillion, M., Herbert, J. S., Pine, J., Vihman, M., DePaolis, R.,
Keren-Portnoy, T., \& Matthews, D. (2017). What paves the way to
conventional language? The predictive value of babble, pointing, and
socioeconomic status. \emph{Child Development}, \emph{88}(1), 156--166.

\hypertarget{ref-mirkovic2016does}{}
Mirković, J., \& Gaskell, M. G. (2016). Does sleep improve your grammar?
Preferential consolidation of arbitrary components of new linguistic
knowledge. \emph{PloS One}, \emph{11}(4), e0152489.
doi:\href{https://doi.org/10.1371/journal.pone.0152489}{10.1371/journal.pone.0152489}

\hypertarget{ref-morelli1992cultural}{}
Morelli, G. A., Rogoff, B., Oppenheim, D., \& Goldsmith, D. (1992).
Cultural variation in infants' sleeping arrangements: Questions of
independence. \emph{Developmental Psychology}, \emph{28}(4), 604.
doi:\href{https://doi.org/10.1037/0012-1649.28.4.604}{10.1037/0012-1649.28.4.604}

\hypertarget{ref-nielsen2017persistent}{}
Nielsen, M., Haun, D., Kärtner, J., \& Legare, C. H. (2017). The
persistent sampling bias in developmental psychology: A call to action.
\emph{Journal of Experimental Child Psychology}, \emph{162}, 31--38.
doi:\href{https://doi.org/10.1016/j.jecp.2017.04.017}{10.1016/j.jecp.2017.04.017}

\hypertarget{ref-ochs1984language}{}
Ochs, E., \& Schieffelin, B. (1984). Language acquisition and
socialization: Three developmental stories and their implications. In R.
A. Schweder \& R. A. LeVine (Eds.), \emph{Culture theory: Essays on
mind, self, and emotion} (pp. 276--322). Cambridge University Press.

\hypertarget{ref-oller1988role}{}
Oller, D. K., \& Eilers, R. E. (1988). The role of audition in infant
babbling. \emph{Child Development}, 441--449.

\hypertarget{ref-oller1995extreme}{}
Oller, D. K., Eilers, R. E., Basinger, D., Steffens, M. L., \& Urbano,
R. (1995). Extreme poverty and the development of precursors to the
speech capacity. \emph{First Language}, \emph{15}(44), 167--187.

\hypertarget{ref-oller1998late}{}
Oller, D. K., Eilers, R. E., Neal, A. R., \& Cobo-Lewis, A. B. (1998).
Late onset canonical babbling: A possible early marker of abnormal
development. \emph{American Journal on Mental Retardation},
\emph{103}(3), 249--263.

\hypertarget{ref-oller1994speech}{}
Oller, D. K., Eilers, R. E., Steffens, M. L., Lynch, M. P., \& Urbano,
R. (1994). Speech-like vocalizations in infancy: An evaluation of
potential risk factors {[}*{]}. \emph{Journal of Child Language},
\emph{21}(1), 33--58.

\hypertarget{ref-pine1993reanalysing}{}
Pine, J. M., \& Lieven, E. V. M. (1993). Reanalysing rote-learned
phrases: Individual differences in the transition to multi-word speech.
\emph{Journal of Child Language}, \emph{20}(3), 551--571.
doi:\href{https://doi.org/10.1017/S0305000900008473}{10.1017/S0305000900008473}

\hypertarget{ref-pye1986quiche}{}
Pye, C. (1986). Quiché Mayan speech to children. \emph{Journal of Child
Language}, \emph{13}(1), 85--100.
doi:\href{https://doi.org/10.1017/S0305000900000313}{10.1017/S0305000900000313}

\hypertarget{ref-pye2017comparative}{}
Pye, C. (2017). \emph{The Comparative Method of Language Acquisition
Research}. University of Chicago Press.

\hypertarget{ref-R-base}{}
R Core Team. (2018). \emph{R: A language and environment for statistical
computing}. Vienna, Austria: R Foundation for Statistical Computing.
Retrieved from \url{https://www.R-project.org/}

\hypertarget{ref-rogoff2003firsthand}{}
Rogoff, B., Paradise, R., Arauz, R. M., Correa-Chávez, M., \& Angelillo,
C. (2003). Firsthand learning through intent participation. \emph{Annual
Review of Psychology}, \emph{54}(1), 175--203.
doi:\href{https://doi.org/10.1146/annurev.psych.54.101601.145118}{10.1146/annurev.psych.54.101601.145118}

\hypertarget{ref-rowe2008child}{}
Rowe, M. L. (2008). Child-directed speech: Relation to socioeconomic
status, knowledge of child development and child vocabulary skill.
\emph{Journal of Child Language}, \emph{35}(1), 185--205.
doi:\href{https://doi.org/10.1017/S0305000907008343}{10.1017/S0305000907008343}

\hypertarget{ref-scaffIPlanguage}{}
Scaff, C., Stieglitz, J., Casillas, M., \& Cristia, A. (in preparation).
Language input in a hunter-forager population: Estimations from daylong
recordings.

\hypertarget{ref-schwab2016repetition}{}
Schwab, J. F., \& Lew-Williams, C. (2016). Repetition across successive
sentences facilitates young children's word learning.
\emph{Developmental Psychology}, \emph{52}(6), 879--886.
doi:\href{https://doi.org/10.1037/dev0000125}{10.1037/dev0000125}

\hypertarget{ref-shneidman2010language}{}
Shneidman, L. A. (2010). \emph{Language Input and Acquisition in a Mayan
Village} (PhD thesis). The University of Chicago.

\hypertarget{ref-shneidman2012language}{}
Shneidman, L. A., \& Goldin-Meadow, S. (2012). Language input and
acquisition in a Mayan village: How important is directed speech?
\emph{Developmental Science}, \emph{15}(5), 659--673.
doi:\href{https://doi.org/10.1111/j.1467-7687.2012.01168.x}{10.1111/j.1467-7687.2012.01168.x}

\hypertarget{ref-slobin1970universals}{}
Slobin, D. I. (1970). Universals of grammatical development in children.
In G. B. Flores d'Arcais \& W. J. M. Levelt (Eds.), \emph{Advances in
Psycholinguistics} (pp. 174--186). Amsterdam, NL: North Holland
Publishing.

\hypertarget{ref-smithson2013generalized}{}
Smithson, M., \& Merkle, E. (2013). \emph{Generalized linear models for
categorical and continuous limited dependent variables}. New York:
Chapman; Hall/CRC.
doi:\href{https://doi.org/10.1201/b15694}{10.1201/b15694}

\hypertarget{ref-soderstrom2007beyond}{}
Soderstrom, M. (2007). Beyond babytalk: Re-evaluating the nature and
content of speech input to preverbal infants. \emph{Developmental
Review}, \emph{27}(4), 501--532.
doi:\href{https://doi.org/10.1016/j.dr.2007.06.002}{10.1016/j.dr.2007.06.002}

\hypertarget{ref-soderstrom2013when}{}
Soderstrom, M., \& Wittebolle, K. (2013). When do caregivers talk? The
influences of activity and time of day on caregiver speech and child
vocalizations in two childcare environments. \emph{PloS One}, \emph{8},
e80646.
doi:\href{https://doi.org/10.1371/journal.pone.0080646}{10.1371/journal.pone.0080646}

\hypertarget{ref-tamislemonda2018routine}{}
Tamis-LeMonda, C. S., Custode, S., Kuchirko, Y., Escobar, K., \& Lo, T.
(2018). Routine language: Speech directed to infants during home
activities. \emph{Child Development}, \emph{Early View}, 1--18.

\hypertarget{ref-tamislemonda2017power}{}
Tamis-LeMonda, C. S., Kuchirko, Y., Luo, R., Escobar, K., \& Bornstein,
M. H. (2017). Power in methods: Language to infants in structured and
naturalistic contexts. \emph{Developmental Science}, \emph{20}(6),
e12456.
doi:\href{https://doi.org/10.1111/desc.12456}{10.1111/desc.12456}

\hypertarget{ref-tomasello1999early}{}
Tomasello, M., \& Brooks, P. J. (1999). Early syntactic development: A
Construction Grammar approach. In M. Barrett (Ed.), \emph{The
Development of Language} (pp. 161--190). New York: Psychology Press.

\hypertarget{ref-vogt2015communicative}{}
Vogt, P., Mastin, J. D., \& Schots, D. M. A. (2015). Communicative
intentions of child-directed speech in three different learning
environments: Observations from the Netherlands, and rural and urban
Mozambique. \emph{First Language}, \emph{35}(4--5), 341--358.
doi:\href{https://doi.org/10.1177/0142723715596647}{10.1177/0142723715596647}

\hypertarget{ref-warlaumont2014social}{}
Warlaumont, A. S., Richards, J. A., Gilkerson, J., \& Oller, D. K.
(2014). A social feedback loop for speech development and its reduction
in Autism. \emph{Psychological Science}, \emph{25}(7), 1314--1324.
doi:\href{https://doi.org/10.1177/0956797614531023}{10.1177/0956797614531023}

\hypertarget{ref-weisleder2013talking}{}
Weisleder, A., \& Fernald, A. (2013). Talking to children matters: Early
language experience strengthens processing and builds vocabulary.
\emph{Psychological Science}, \emph{24}(11), 2143--2152.
doi:\href{https://doi.org/10.1177/0956797613488145}{10.1177/0956797613488145}

\hypertarget{ref-R-ggplot2}{}
Wickham, H. (2009). \emph{Ggplot2: Elegant graphics for data analysis}.
Springer-Verlag New York. Retrieved from \url{http://ggplot2.org}

\hypertarget{ref-ELAN}{}
Wittenburg, P., Brugman, H., Russel, A., Klassmann, A., \& Sloetjes, H.
(2006). ELAN: A professional framework for multimodality research. In
\emph{Proceedings of the Fifth International Conference on Language
Resources and Evaluation} (pp. 1556--1559).

\hypertarget{ref-yurovsky2018communicative}{}
Yurovsky, D. (2018). A communicative approach to early word learning.
\emph{New Ideas in Psychology}, \emph{50}, 73--79.
doi:\href{https://doi.org/10.1016/j.newideapsych.2017.09.001}{10.1016/j.newideapsych.2017.09.001}

\endgroup






\end{document}
