\documentclass[floatsintext,man]{apa6}

\usepackage{amssymb,amsmath}
\usepackage{ifxetex,ifluatex}
\usepackage{fixltx2e} % provides \textsubscript
\ifnum 0\ifxetex 1\fi\ifluatex 1\fi=0 % if pdftex
  \usepackage[T1]{fontenc}
  \usepackage[utf8]{inputenc}
\else % if luatex or xelatex
  \ifxetex
    \usepackage{mathspec}
    \usepackage{xltxtra,xunicode}
  \else
    \usepackage{fontspec}
  \fi
  \defaultfontfeatures{Mapping=tex-text,Scale=MatchLowercase}
  \newcommand{\euro}{€}
\fi
% use upquote if available, for straight quotes in verbatim environments
\IfFileExists{upquote.sty}{\usepackage{upquote}}{}
% use microtype if available
\IfFileExists{microtype.sty}{\usepackage{microtype}}{}

% Table formatting
\usepackage{longtable, booktabs}
\usepackage{lscape}
% \usepackage[counterclockwise]{rotating}   % Landscape page setup for large tables
\usepackage{multirow}		% Table styling
\usepackage{tabularx}		% Control Column width
\usepackage[flushleft]{threeparttable}	% Allows for three part tables with a specified notes section
\usepackage{threeparttablex}            % Lets threeparttable work with longtable

% Create new environments so endfloat can handle them
% \newenvironment{ltable}
%   {\begin{landscape}\begin{center}\begin{threeparttable}}
%   {\end{threeparttable}\end{center}\end{landscape}}

\newenvironment{lltable}
  {\begin{landscape}\begin{center}\begin{ThreePartTable}}
  {\end{ThreePartTable}\end{center}\end{landscape}}




% The following enables adjusting longtable caption width to table width
% Solution found at http://golatex.de/longtable-mit-caption-so-breit-wie-die-tabelle-t15767.html
\makeatletter
\newcommand\LastLTentrywidth{1em}
\newlength\longtablewidth
\setlength{\longtablewidth}{1in}
\newcommand\getlongtablewidth{%
 \begingroup
  \ifcsname LT@\roman{LT@tables}\endcsname
  \global\longtablewidth=0pt
  \renewcommand\LT@entry[2]{\global\advance\longtablewidth by ##2\relax\gdef\LastLTentrywidth{##2}}%
  \@nameuse{LT@\roman{LT@tables}}%
  \fi
\endgroup}


\ifxetex
  \usepackage[setpagesize=false, % page size defined by xetex
              unicode=false, % unicode breaks when used with xetex
              xetex]{hyperref}
\else
  \usepackage[unicode=true]{hyperref}
\fi
\hypersetup{breaklinks=true,
            pdfauthor={},
            pdftitle={Child language experience in a Tseltal Mayan village},
            colorlinks=true,
            citecolor=blue,
            urlcolor=blue,
            linkcolor=black,
            pdfborder={0 0 0}}
\urlstyle{same}  % don't use monospace font for urls

\setlength{\parindent}{0pt}
%\setlength{\parskip}{0pt plus 0pt minus 0pt}

\setlength{\emergencystretch}{3em}  % prevent overfull lines


% Manuscript styling
\captionsetup{font=singlespacing,justification=justified}
\usepackage{csquotes}
\usepackage{upgreek}

 % Line numbering
  \usepackage{lineno}
  \linenumbers


\usepackage{tikz} % Variable definition to generate author note

% fix for \tightlist problem in pandoc 1.14
\providecommand{\tightlist}{%
  \setlength{\itemsep}{0pt}\setlength{\parskip}{0pt}}

% Essential manuscript parts
  \title{Child language experience in a Tseltal Mayan village}

  \shorttitle{Child language experience in a Tseltal Mayan village}


  \author{Marisa Casillas\textsuperscript{1}, Penelope Brown\textsuperscript{1}, \& Stephen C. Levinson\textsuperscript{1}}

  % \def\affdep{{"", "", ""}}%
  % \def\affcity{{"", "", ""}}%

  \affiliation{
    \vspace{0.5cm}
          \textsuperscript{1} Max Planck Institute for Psycholinguistics  }

  \authornote{
    Correspondence concerning this article should be addressed to Marisa
    Casillas, P.O. Box 310, 6500 AH Nijmegen, The Netherlands. E-mail:
    \href{mailto:Marisa.Casillas@mpi.nl}{\nolinkurl{Marisa.Casillas@mpi.nl}}
  }


  \abstract{We analyzed 9--11-hour at-home audio recordings from 10 Tseltal Mayan
children between 0;2 and 3;0 to investigate how often they engaged in
verbal interaction with others and whether their speech environment
changed with age, time of day, household size, and number of speakers
present. We found that Tseltal children are not often directly spoken
to, that most directed speech comes from adults, and that directed
speech does not increase with age. Most of children's directed speech
came in the mornings or early evenings, particularly with younger
children, and high interactional peaks tended to occur in bursts of turn
taking that lasted approximately one minute. With some exceptions, these
findings support previous characterizations of Mayan caregiver-child
talk. An initial analysis of children's vocal development suggests that,
despite relatively little directed speech, these children develop early
language skills on a similar timescale to WEIRD children. Given these
findings, we discuss multiple proposals for how Tseltal children might
be efficient learners.}
  \keywords{Child-directed speech, linguistic input, non-WEIRD, vocal maturity, turn
taking, interaction, Mayan \\

    \indent Word count: X
  }





\usepackage{amsthm}
\newtheorem{theorem}{Theorem}[section]
\newtheorem{lemma}{Lemma}[section]
\theoremstyle{definition}
\newtheorem{definition}{Definition}[section]
\newtheorem{corollary}{Corollary}[section]
\newtheorem{proposition}{Proposition}[section]
\theoremstyle{definition}
\newtheorem{example}{Example}[section]
\theoremstyle{definition}
\newtheorem{exercise}{Exercise}[section]
\theoremstyle{remark}
\newtheorem*{remark}{Remark}
\newtheorem*{solution}{Solution}
\begin{document}

\maketitle

\setcounter{secnumdepth}{0}



\section{Introduction}\label{intro}

A great deal of work in developmental language science revolves around
one central question: What linguistic evidence is needed to support
first language acquisition? In pursuing this topic, many researchers
have fixed their sights on the quantity and characteristics of speech
addressed to children; that is, speech designed for young recipients who
may have limited attention and understanding (e.g., Golinkoff, Can,
Soderstrom, \& Hirsh-Pasek, 2015; Hoff, 2006). In several languages,
child-directed speech (CDS\footnote{Throughout this article, we use
  \enquote{child-directed speech} and \enquote{CDS} in the most literal
  sense: speech designed for and directed toward a child recipient.})
has been demonstrated to be distinct from adult-directed speech (ADS) in
that it is linguistically adapted for young listeners (Cristia, 2013;
Soderstrom, 2007), interactionally rich (Bruner, 1983; Butterworth,
2003; Estigarribia \& Clark, 2007; Masataka, 2003), and preferred by
infants (Cooper \& Aslin, 1990; ManyBabies Collaborative, 2017; Segal \&
Newman, 2015). In those same linguistic communities, these properties of
CDS have been found to facilitate early word learning (e.g., Cartmill et
al., 2013; Hirsh-Pasek et al., 2015; Hoff, 2003; Hurtado, Marchman, \&
Fernald, 2008; Rowe, 2008; Shneidman \& Goldin-Meadow, 2012; Shneidman,
Arroyo, Levine, \& Goldin-Meadow, 2012; Weisleder \& Fernald, 2013).
However, ethnographic reports from a number of traditional, non-Western
communities suggest that children easily acquire their community's
language(s) even when they are only infrequently directly addressed (P.
Brown, 2011). If so, frequent CDS may not be essential for learning
language; just useful for facilitating certain aspects of language
development. In this paper we investigate the language environment and
early development of 10 Tseltal Mayan children growing up in a community
where caregivers have been reported to infrequently directly address
speech to infants and young children (P. Brown, 1998, 2011, 2014).

\subsection{Child-directed speech}\label{intro-cds}

Prior work on CDS in Western contexts has shown that the amount of CDS
children hear influences their language development; more CDS is
associated with larger and faster-growing receptive and productive
vocabularies (e.g., Hart \& Risley, 1995; Hoff, 2003; Hurtado et al.,
2008; Peter, Durrant, Bidgood, Pine, \& Rowland, in preparation;
Ramírez-Esparza, García-Sierra, \& Kuhl, 2014, 2017; Shneidman \&
Goldin-Meadow, 2012; Shneidman et al., 2012; Weisleder \& Fernald,
2013). CDS has also been linked to young children's speed of lexical
retrieval (Hurtado et al., 2008; Weisleder \& Fernald, 2013; but see
Peter et al., in preparation) and syntactic development (Huttenlocher,
Waterfall, Vasilyeva, Vevea, \& Hedges, 2010). The conclusion drawn from
much of this work is that speech directed to children is well designed
for learning words---especially concrete nouns---because it is optimized
for a child's attention in the moment the words are uttered. Indeed,
infants and young children prefer listening to attention-grabbing CDS
over ADS, even outside of first-person interaction (ManyBabies
Collaborative, 2017). There are, however, a few significant caveats to
this body of work relating CDS quantity to language development.

First, while there is overwhelming evidence linking CDS quantity to
vocabulary size, links to grammatical development are more scant (e.g.,
Brinchmann, Braeken, \& Lyster, 2019; Frank, Braginsky, Marchman, \&
Yurovsky, in preparation; Huttenlocher et al., 2010). While the
advantage of CDS for referential word learning is clear, it is less
obvious how CDS facilitates syntactic learning. For example, utterance
length (a proxy for syntactic complexity; Wasow, 1997) doesn't appear to
increase with child age (Newport, Gleitman, \& Gleitman, 1977), and
parents may be less likely to directly correct their children's
syntactic errors than their semantic ones (R. Brown, 1977; but see
Chouinard \& Clark, 2003)---sometimes themselves producing ungrammatical
utterances to make individual words salient (Aslin, Woodward, LaMendola,
\& Bever, 1996). On the other hand, there is a wealth of evidence that
syntactic knowledge is lexically specified (e.g., Arnold, Wasow, Asudeh,
\& Alrenga, 2004; Goldberg, 2003; Lieven, Pine, \& Baldwin, 1997), and
that, crosslinguistically, children's vocabulary size is one of the most
robust predictors of their early syntactic development (Bates \&
Goodman, 1997; Frank et al., in preparation; Marchman,
Martínez-Sussmann, \& Dale, 2004). In short, what is good for the
lexicon may also be good for syntax. For now, however, the direct link
between CDS and grammatical development still needs further exploration
(see also Yurovsky, 2018).

A second caveat is that most work on CDS quantity uses summary measures
that average over the ebb and flow of interaction (e.g., proportion
CDS). In both child and adult interactions, verbal behaviors are highly
structured: while some occur at fairly regular intervals
(\enquote{periodic}), others occur in shorter, more intense bouts
separated by long periods of inactivity (``bursty''; Abney, Dale,
Louwerse, \& Kello, 2018; Fusaroli, Razczaszek-Leonardi, \& Tylén,
2014). For example, Abney and colleagues (2017) found that, across
multiple time scales of daylong recordings, both infants' and adults'
vocal behavior was clustered. Focusing specifically on lexical
development, Blasi and colleagues (in preparation) found that nouns and
verbs were used burstily in child-proximal speech across all six of the
languages in their typologically diverse sample. Infrequent words were
somewhat more bursty overall, leading them to propose that burstiness
may play a key and universal role in acquiring otherwise-rare linguistic
units. Experiment-based work also shows that two-year-olds learn novel
words better from a massed presentation of object labels versus a
distributed presentation (Schwab \& Lew-Williams, 2016; but see
Ambridge, Theakston, Lieven, \& Tomasello, 2006 and Childers \&
Tomasello, 2002). These structured temporal characteristics in
children's language experience imply new roles for attention and memory
in language development. Ideally, then, we should be investigating how
CDS is distributed over children's daily experiences (Soderstrom \&
Wittebolle, 2013).

Finally, prior work has typically focused on Western (primarily North
American) populations, limiting our ability to generalize these effects
to children acquiring language worldwide (P. Brown \& Gaskins, 2014;
Henrich, Heine, \& Norenzayan, 2010; Lieven, 1994; Nielsen, Haun,
Kärtner, \& Legare, 2017). While we do gain valuable insight by looking
at within-population variation (e.g., different socioeconomic or
sub-cultures), we can more effectively find places where our assumptions
break down by studying new populations. Linguistic anthropologists
working in non-Western communities have long reported that caregiver
interaction styles vary immensely from place to place, with some
caregivers using little child-directed speech with young children (P.
Brown \& Gaskins, 2014; Gaskins, 2006; Lieven, 1994; Ochs \&
Schieffelin, 1984). Children in these communities reportedly acquire
language with \enquote{typical}-looking benchmarks. For example, they
start pointing and talking around the same time we would expect for
Western middle-class infants (P. Brown, 2011, 2014; P. Brown \& Gaskins,
2014; Liszkowski, Brown, Callaghan, Takada, \& De Vos, 2012; but see
Salomo \& Liszkowski, 2013). These findings have had little impact on
mainstream theories of word learning and language acquisition, partly
due to a lack of directly comparable measures (P. Brown, 2014; P. Brown
\& Gaskins, 2014). If, however, children in these communities do acquire
language without delay, despite infrequent CDS, we must reconsider what
kind of linguistic evidence is necessary for children to learn language.

\subsection{Language development in non-WEIRD
communities}\label{intro-nonweird}

A growing number of researchers are using methods from developmental
psycholinguistics to describe the language environments and linguistic
development of children growing up in traditional and/or non-Western
communities (see also Barrett et al., 2013; Demuth, Moloi, \& Machobane,
2010; Fortier, Kellier, Fernández Flecha, \& Frank, under review; Ganek,
Smyth, Nixon, \& Eriks-Brophy, 2018; Garcia, Roeser, \& Höhle, 2018;
Hernik \& Broesch, 2018). We briefly highlight two recent efforts along
these lines, but see Cristia and colleagues' (2017) and Mastin and
Vogt's work (2016; 2015) for similar examples.

Scaff, Cristia, and colleagues (2017; in preparation) have used a number
of methods to estimate how much speech children hear in a Tsimane
forager-horticulturalist population in the Bolivian lowlands. From
daylong audio recordings, they estimate that Tsimane children between
0;6 and 6;0 hear maximally \textasciitilde{}4.8 minutes of directly
addressed speech per hour, regardless of their age (Cristia et al.,
2017; Scaff et al., in preparation). For comparison, children from North
American homes between ages 0;3 and 3;0 are estimated to hear
\textasciitilde{}11 minutes of CDS per hour in daylong recordings
(Bergelson et al., 2019). Note however, that these estimates from from a
non-random sample of clips that were selected based on the presence of
adult speech.

Shneidman and colleagues (2010; 2012) analyzed speech from one-hour
at-home video recordings of children between ages 1;0 and 3;0 in two
communities: Yucatec Mayan (Southern Mexico) and North American (a major
U.S. city). Their analyses yielded four main findings: compared to the
American children, (a) the Yucatec children heard many fewer utterances
per hour, (b) a much smaller proportion of the utterances they heard
were child-directed, (c) the proportion of utterances that were
child-directed increased dramatically with age, matching U.S. children's
CDS proportion by 3;0, and (d) most of the added CDS came from other
children (e.g., older siblings and cousins). They also demonstrated that
the lexical diversity of the CDS they hear at 24 months---particularly
from adult speakers---predicted children's vocabulary knowledge at 35
months.

\subsection{The current study}\label{intro-currentstudy}

We examine the early language experience of 10 Tseltal Mayan children
under age 3;0. Prior ethnographic work suggests that Tseltal caregivers
do not frequently directly speak to their children until the children
themselves begin to actively initiate verbal interactions (P. Brown,
2011, 2014). Nonetheless, Tseltal children develop language with no
apparent delays. Tseltal Mayan language and culture has much in common
with the Yucatec Mayan communities Shneidman (2010; 2012) reports
on.\footnote{For a review of comparative work on language socialization
  in Mayan cultures, see Pye (2017).} We provide more details on this
community and dataset in the \protect\hyperlink{methods}{Methods
section}.

We analyzed basic measures of Tseltal children's language environments
including: (a) the quantity of speech directed to them, (b) the quantity
of other-directed speech they could potentially overhear from nearby
speakers, (c) the rate of contingent responses to their vocalizations,
(d) the rate of their contingent responses to others' vocalizations, and
(e) the duration of their interactional dyadic sequences. We link these
findings to prior work on speech environment and development, and
roughly estimated the number of minutes per day children spent in
\enquote{high turn-taking} interaction. We also outline a basic
trajectory for early vocal development (i.e., from non-canonical babbles
to multi-word utterances).

Based on prior work, we predicted that Tseltal Mayan children are
infrequently directly addressed, that the amount of CDS and contingent
responses they hear increases with age, that most CDS comes from other
children, and that, despite this, their early vocal development is on
par with Western children. We additionally predicted that children's
language environments would be bursty---that high-intensity interactions
would be brief and sparsely distributed throughout the day, accounting
for the majority of children's daily CDS.

\section{References}\label{refs}

\begingroup
\setlength{\parindent}{-0.5in} \setlength{\leftskip}{0.5in}

\hypertarget{refs}{}
\hypertarget{ref-abney2018bursts}{}
Abney, D. H., Dale, R., Louwerse, M. M., \& Kello, C. T. (2018). The
bursts and lulls of multimodal interaction: Temporal distributions of
behavior reveal differences between verbal and non-verbal communication.
\emph{Cognitive Science}, \emph{XX}, XX--XX.
doi:\href{https://doi.org/10.1111/cogs.12612}{10.1111/cogs.12612}

\hypertarget{ref-abney2017time}{}
Abney, D. H., Smith, L. B., \& Yu, C. (2017). It's time: Quantifying the
relevant time scales for joint attention. In G. Gunzelmann, A. Howes, T.
Tenbrink, \& E. Davelaar (Eds.), \emph{Proceedings of the 39th annual
meeting of the cognitive science society} (pp. 1489--1494). London, UK.

\hypertarget{ref-arnold2004avoiding}{}
Arnold, J. E., Wasow, T., Asudeh, A., \& Alrenga, P. (2004). Avoiding
attachment ambiguities: The role of constituent ordering. \emph{Journal
of Memory and Language}, \emph{51}(1), 55--70.

\hypertarget{ref-aslin1996models}{}
Aslin, R. N., Woodward, J. Z., LaMendola, N. P., \& Bever, T. G. (1996).
Models of word segmentation in fluent maternal speech to infants. In J.
L. Morgan \& K. Demuth (Eds.), \emph{Signal to syntax: Bootstrapping
from speech to grammar in early acquisition} (pp. 117--134). New York,
NY: Psychology Press.

\hypertarget{ref-barrett2013early}{}
Barrett, H., Broesch, T., Scott, R. M., He, Z., Baillargeon, R., Wu, D.,
\ldots{} Stephen Laurence. (2013). Early false-belief understanding in
traditional non-Western societies. \emph{Proceedings of the Royal
Society B: Biological Sciences}, \emph{280}(1755), XX--XX.
doi:\href{https://doi.org/10.1098/rspb.2012.2654}{10.1098/rspb.2012.2654}

\hypertarget{ref-bates1997inseparability}{}
Bates, E., \& Goodman, J. C. (1997). On the inseparability of grammar
and the lexicon: Evidence from acquisition, aphasia, and real-time
processing. \emph{Language and Cognitive Processes}, \emph{12}(5--6),
507--584.
doi:\href{https://doi.org/doi.org/10.1080/016909697386628}{doi.org/10.1080/016909697386628}

\hypertarget{ref-bergelsoncasillas2019what}{}
Bergelson, E., Casillas, M., Soderstrom, M., Seidl, A., Warlaumont, A.
S., \& Amatuni, A. (2019). What do north american babies hear? A
large-scale cross-corpus analysis. \emph{Developmental Science},
\emph{22}(1), e12724.
doi:\href{https://doi.org/doi:10.1111/desc.12724}{doi:10.1111/desc.12724}

\hypertarget{ref-blasiIPhuman}{}
Blasi, D., Schikowski, R., Moran, S., Pfeiler, B., \& Stoll, S. (in
preparation). Human communication is structured efficiently for first
language learners: Lexical spikes.

\hypertarget{ref-brinchmann2019direct}{}
Brinchmann, E. I., Braeken, J., \& Lyster, S.-A. H. (2019). Is there a
direct relation between the development of vocabulary and grammar?
\emph{Developmental Science}, \emph{22}(1), e12709.

\hypertarget{ref-brown1998conversational}{}
Brown, P. (1998). Conversational structure and language acquisition: The
role of repetition in tzeltal adult and child speech. \emph{Journal of
Linguistic Anthropology}, \emph{2}, 197--221.
doi:\href{https://doi.org/10.1525/jlin.1998.8.2.197}{10.1525/jlin.1998.8.2.197}

\hypertarget{ref-brown2011cultural}{}
Brown, P. (2011). The cultural organization of attention. In A. Duranti,
E. Ochs, \& and B.B. Schieffelin (Eds.), \emph{Handbook of language
socialization} (pp. 29--55). Malden, MA: Wiley-Blackwell.

\hypertarget{ref-brown2014interactional}{}
Brown, P. (2014). The interactional context of language learning in
tzeltal. In I. Arnon, M. Casillas, C. Kurumada, \& B. Estigarribia
(Eds.), \emph{Language in interaction: Studies in honor of eve v. clark}
(pp. 51--82). Amsterdam, NL: John Benjamins.

\hypertarget{ref-brown2014language}{}
Brown, P., \& Gaskins, S. (2014). Language acquisition and language
socialization. In N. J. Enfield, P. Kockelman, \& J. Sidnell (Eds.),
\emph{Handbook of linguistic anthropology} (pp. 183--222). Cambridge,
UK: Cambridge University Press.

\hypertarget{ref-brown1997introduction}{}
Brown, R. (1977). Introduction. In C. E. Snow \& C. A. Ferguson (Eds.),
\emph{Talking to children: Language input and interaction} (pp. 1--30).
Cambridge, UK: Cambridge University Press.

\hypertarget{ref-bruner1983childs}{}
Bruner, J. (1983). \emph{Child's talk}. Oxford: Oxford University Press.

\hypertarget{ref-butterworth2003pointing}{}
Butterworth, G. (2003). Pointing is the royal road to language for
babies. In S. Kita (Ed.), \emph{Pointing} (pp. 17--42). Psychology
Press.

\hypertarget{ref-cartmill2013quality}{}
Cartmill, E. A., Armstrong, B. F., Gleitman, L. R., Goldin-Meadow, S.,
Medina, T. N., \& Trueswell, J. C. (2013). Quality of early parent input
predicts child vocabulary 3 years later. \emph{Proceedings of the
National Academy of Sciences}, \emph{110}(28), 11278--11283.

\hypertarget{ref-chouinard2003adult}{}
Chouinard, M. M., \& Clark, E. V. (2003). Adult reformulations of child
errors as negative evidence. \emph{Journal of Child Language},
\emph{30}(3), 637--669.

\hypertarget{ref-cooper1990preference}{}
Cooper, R. P., \& Aslin, R. N. (1990). Preference for infant-directed
speech in the first month after birth. \emph{Child Development},
\emph{20}(4), 477--488.
doi:\href{https://doi.org/10.1016/S0163-6383(97)90037-0}{10.1016/S0163-6383(97)90037-0}

\hypertarget{ref-cristia2013input}{}
Cristia, A. (2013). Input to language: The phonetics and perception of
infant-directed speech. \emph{Language and Linguistics Compass},
\emph{7}(3), 157--170.
doi:\href{https://doi.org/10.1111/lnc3.12015}{10.1111/lnc3.12015}

\hypertarget{ref-cristia2017child}{}
Cristia, A., Dupoux, E., Gurven, M., \& Stieglitz, J. (2017).
Child-directed speech is infrequent in a forager-farmer population: A
time allocation study. \emph{Child Development}, XX--XX.
doi:\href{https://doi.org/10.1111/cdev.12974}{10.1111/cdev.12974}

\hypertarget{ref-demuth2010three}{}
Demuth, K., Moloi, F., \& Machobane, M. (2010). 3-year-olds'
comprehension, production, and generalization of Sesotho passives.
\emph{Cognition}, \emph{115}(2), 238--251.

\hypertarget{ref-estigarribia2007getting}{}
Estigarribia, B., \& Clark, E. V. (2007). Getting and maintaining
attention in talk to young children. \emph{Journal of Child Language},
\emph{34}(4), 799--814.

\hypertarget{ref-fortierURadhoc}{}
Fortier, M. E., Kellier, D., Fernández Flecha, M., \& Frank, M. C.
(under review). Ad-hoc pragmatic implicatures among Shipibo-Konibo
children in the Peruvian Amazon.
doi:\href{https://doi.org/10.31234/osf.io/x7ad9}{10.31234/osf.io/x7ad9}

\hypertarget{ref-frankIPvariability}{}
Frank, M. C., Braginsky, M., Marchman, V. A., \& Yurovsky, D. (in
preparation). \emph{Variability and consistency in early language
learning: The Wordbank project}. XX. Retrieved from
\url{https://langcog.github.io/wordbank-book/}

\hypertarget{ref-fusaroli2014synergy}{}
Fusaroli, R., Razczaszek-Leonardi, J., \& Tylén, K. (2014). Dialog as
interpersonal synergy. \emph{New Ideas in Psychology}, \emph{32},
147--157.
doi:\href{https://doi.org/10.1016/j.newideapsych.2013.03.005}{10.1016/j.newideapsych.2013.03.005}

\hypertarget{ref-ganek2018using}{}
Ganek, H., Smyth, R., Nixon, S., \& Eriks-Brophy, A. (2018). Using the
Language ENvironment analysis (LENA) system to investigate cultural
differences in conversational turn count. \emph{Journal of Speech,
Language, and Hearing Research}, \emph{61}, 2246--2258.
doi:\href{https://doi.org/10.1044/2018_JSLHR-L-17-0370}{10.1044/2018\_JSLHR-L-17-0370}

\hypertarget{ref-garcia2018thematic}{}
Garcia, R., Roeser, J., \& Höhle, B. (2018). Thematic role assignment in
the L1 acquisition of Tagalog: Use of word order and morphosyntactic
markers. \emph{Language Acquisition}, \emph{XX}, XX--XX.
doi:\href{https://doi.org/10.1080/10489223.2018.1525613}{10.1080/10489223.2018.1525613}

\hypertarget{ref-gaskins2006cultural}{}
Gaskins, S. (2006). Cultural perspectives on infant--caregiver
interaction. In N. J. Enfield \& S. Levinson (Eds.), \emph{Roots of
human sociality: Culture, cognition and interaction} (pp. 279--298).
Oxford: Berg.

\hypertarget{ref-goldberg2003constructions}{}
Goldberg, A. E. (2003). Constructions: A new theoretical approach to
language. \emph{Trends in Cognitive Sciences}, \emph{7}(5), 219--224.

\hypertarget{ref-golinkoff2015baby}{}
Golinkoff, R. M., Can, D. D., Soderstrom, M., \& Hirsh-Pasek, K. (2015).
(Baby) talk to me: The social context of infant-directed speech and its
effects on early language acquisition. \emph{Current Directions in
Psychological Science}, \emph{24}(5), 339--344.

\hypertarget{ref-hart1995meaningful}{}
Hart, B., \& Risley, T. R. (1995). \emph{Meaningful Differences in the
Everyday Experience of Young American Children}. Paul H. Brookes
Publishing.

\hypertarget{ref-henrich2010beyond}{}
Henrich, J., Heine, S. J., \& Norenzayan, A. (2010). Beyond WEIRD:
Towards a broad-based behavioral science. \emph{Behavioral and Brain
Sciences}, \emph{33}(2--3), 111--135.

\hypertarget{ref-hernik2018infant}{}
Hernik, M., \& Broesch, T. (2018). Infant gaze following depends on
communicative signals: An eye-tracking study of 5- to 7-month-olds in
Vanuatu. \emph{Developmental Science}, \emph{XX}, XX--XX.
doi:\href{https://doi.org/10.1111/desc.12779}{10.1111/desc.12779}

\hypertarget{ref-hirshpasek2015contribution}{}
Hirsh-Pasek, K., Adamson, L. B., Bakeman, R., Owen, M. T., Golinkoff, R.
M., Pace, A., \ldots{} Suma, K. (2015). The contribution of early
communication quality to low-income children's language success.
\emph{Psychological Science}, \emph{26}(7), 1071--1083.
doi:\href{https://doi.org/10.1177/0956797615581493}{10.1177/0956797615581493}

\hypertarget{ref-hoff2003specificity}{}
Hoff, E. (2003). The specificity of environmental influence:
Socioeconomic status affects early vocabulary development via maternal
speech. \emph{Child Development}, \emph{74}(5), 1368--1378.

\hypertarget{ref-hoff2006social}{}
Hoff, E. (2006). How social contexts support and shape language
development. \emph{Developmental Review}, \emph{26}(1), 55--88.

\hypertarget{ref-hurtado2008does}{}
Hurtado, N., Marchman, V. A., \& Fernald, A. (2008). Does input
influence uptake? Links between maternal talk, processing speed and
vocabulary size in spanish-learning children. \emph{Developmental
Science}, \emph{11}(6), F31--F39.
doi:\href{https://doi.org/10.1111/j.1467-7687.2008.00768.x\%5D}{10.1111/j.1467-7687.2008.00768.x{]}}

\hypertarget{ref-huttenlocher2010sources}{}
Huttenlocher, J., Waterfall, H., Vasilyeva, M., Vevea, J., \& Hedges, L.
V. (2010). Sources of variability in children's language growth.
\emph{Cognitive Psychology}, \emph{61}, 343--365.

\hypertarget{ref-lieven1994crosslinguistic}{}
Lieven, E. V. M. (1994). Crosslinguistic and crosscultural aspects of
language addressed to children. In C. Gallaway \& B. J. Richards (Eds.),
\emph{Input and interaction in language acquisition} (pp. 56--73). New
York, NY, US: Cambridge University Press.
doi:\href{https://doi.org/10.1017/CBO9780511620690.005}{10.1017/CBO9780511620690.005}

\hypertarget{ref-lieven1997lexically}{}
Lieven, E. V. M., Pine, J. M., \& Baldwin, G. (1997). Lexically-based
learning and early grammatical development. \emph{Journal of Child
Language}, \emph{24}(1), 187--219.

\hypertarget{ref-liszkowski2012prelinguistic}{}
Liszkowski, U., Brown, P., Callaghan, T., Takada, A., \& De Vos, C.
(2012). A prelinguistic gestural universal of human communication.
\emph{Cognitive Science}, \emph{36}(4), 698--713.

\hypertarget{ref-manybabies2017}{}
ManyBabies Collaborative. (2017). Quantifying sources of variability in
infancy research using the infant-directed speech preference.
\emph{Advances in Methods and Practices in Psychological Science},
\emph{XX}, XX--XX.
doi:\href{https://doi.org/10.31234/osf.io/s98ab}{10.31234/osf.io/s98ab}

\hypertarget{ref-marchman2004language}{}
Marchman, V. A., Martínez-Sussmann, C., \& Dale, P. S. (2004). The
language-specific nature of grammatical development: Evidence from
bilingual language learners. \emph{Developmental Science}, \emph{7}(2),
212--224.

\hypertarget{ref-masataka2003onset}{}
Masataka, N. (2003). \emph{The onset of language}. Cambridge University
Press.

\hypertarget{ref-mastin2016infant}{}
Mastin, J. D., \& Vogt, P. (2016). Infant engagement and early
vocabulary development: A naturalistic observation study of Mozambican
infants from 1;1 to 2;1. \emph{Journal of Child Language}, \emph{43}(2),
235--264.
doi:\href{https://doi.org/0.1017/S0305000915000148}{0.1017/S0305000915000148}

\hypertarget{ref-newport1977mother}{}
Newport, E. L., Gleitman, H., \& Gleitman, L. R. (1977). Mother, i'd
rather do it myself: Some effects and non-effects of maternal speech
style. In C. E. Snow \& C. A. Ferguson (Eds.), \emph{Talking to
children: Language input and interaction} (pp. 109--150). Cambridge, UK:
Cambridge University Press.

\hypertarget{ref-nielsen2017persistent}{}
Nielsen, M., Haun, D., Kärtner, J., \& Legare, C. H. (2017). The
persistent sampling bias in developmental psychology: A call to action.
\emph{Journal of Experimental Child Psychology}, \emph{162}, 31--38.

\hypertarget{ref-ochs1984language}{}
Ochs, E., \& Schieffelin, B. (1984). Language acquisition and
socialization: Three developmental stories and their implications. In R.
A. Schweder \& R. A. LeVine (Eds.), \emph{Culture theory: Essays on
mind, self, and emotion} (pp. 276--322). Cambridge University Press.

\hypertarget{ref-peterIPindividual}{}
Peter, M., Durrant, S., Bidgood, A., Pine, J., \& Rowland, C. (in
preparation). Individual differences in speed of language processing and
its relationship with language development. \emph{XX}, (XX), XX--XX.

\hypertarget{ref-pye2017comparative}{}
Pye, C. (2017). \emph{The comparative method of language acquisition
research}. University of Chicago Press.

\hypertarget{ref-ramirezesparza2014look}{}
Ramírez-Esparza, N., García-Sierra, A., \& Kuhl, P. K. (2014). Look
who's talking: Speech style and social context in language input to
infants are linked to concurrent and future speech development.
\emph{Developmental Science}, \emph{17}, 880--891.

\hypertarget{ref-ramirezesparza2017look}{}
Ramírez-Esparza, N., García-Sierra, A., \& Kuhl, P. K. (2017). Look
who's talking NOW! Parentese speech, social context, and language
development across time. \emph{Frontiers in Psychology}, \emph{8}, 1008.

\hypertarget{ref-rowe2008child}{}
Rowe, M. L. (2008). Child-directed speech: Relation to socioeconomic
status, knowledge of child development and child vocabulary skill.
\emph{Journal of Child Language}, \emph{35}(1), 185--205.

\hypertarget{ref-salomo2013sociocultural}{}
Salomo, D., \& Liszkowski, U. (2013). Sociocultural settings influence
the emergence of prelinguistic deictic gestures. \emph{Child
Development}, \emph{84}(4), 1296--1307.

\hypertarget{ref-scaffIPlanguage}{}
Scaff, C., Stieglitz, J., Casillas, M., \& Cristia, A. (in preparation).
Language input in a hunter-forager population: Estimations from daylong
recordings.

\hypertarget{ref-segal2015infant}{}
Segal, J., \& Newman, R. S. (2015). Infant preferences for structural
and prosodic properties of infant-directed speech in the second year of
life. \emph{Infancy}, \emph{20}(3), 339--351.
doi:\href{https://doi.org/10.1111/infa.12077}{10.1111/infa.12077}

\hypertarget{ref-shneidman2010language}{}
Shneidman, L. A. (2010). \emph{Language input and acquisition in a Mayan
village} (PhD thesis). The University of Chicago.

\hypertarget{ref-shneidman2012language}{}
Shneidman, L. A., \& Goldin-Meadow, S. (2012). Language input and
acquisition in a Mayan village: How important is directed speech?
\emph{Developmental Science}, \emph{15}(5), 659--673.

\hypertarget{ref-shneidman2012counts}{}
Shneidman, L. A., Arroyo, M. E., Levine, S. C., \& Goldin-Meadow, S.
(2012). What counts as effective input for word learning? \emph{Journal
of Child Language}, \emph{40}(3), 672--686.

\hypertarget{ref-soderstrom2007beyond}{}
Soderstrom, M. (2007). Beyond babytalk: Re-evaluating the nature and
content of speech input to preverbal infants. \emph{Developmental
Review}, \emph{27}(4), 501--532.

\hypertarget{ref-soderstrom2013when}{}
Soderstrom, M., \& Wittebolle, K. (2013). When do caregivers talk? The
influences of activity and time of day on caregiver speech and child
vocalizations in two childcare environments. \emph{PloS One}, \emph{8},
e80646.

\hypertarget{ref-vogt2015communicative}{}
Vogt, P., Mastin, J. D., \& Schots, D. M. A. (2015). Communicative
intentions of child-directed speech in three different learning
environments: Observations from the netherlands, and rural and urban
mozambique. \emph{First Language}, \emph{35}(4-5), 341--358.

\hypertarget{ref-wasow1997remarks}{}
Wasow, T. (1997). Remarks on grammatical weight. \emph{Language
Variation and Change}, \emph{9}(1), 81--105.

\hypertarget{ref-weisleder2013talking}{}
Weisleder, A., \& Fernald, A. (2013). Talking to children matters: Early
language experience strengthens processing and builds vocabulary.
\emph{Psychological Science}, \emph{24}(11), 2143--2152.

\hypertarget{ref-yurovsky2018communicative}{}
Yurovsky, D. (2018). A communicative approach to early word learning.
\emph{New Ideas in Psychology}, \emph{50}, 73--79.

\endgroup






\end{document}
